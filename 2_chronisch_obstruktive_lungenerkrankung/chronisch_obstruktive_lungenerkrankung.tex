\ifpdf
    \graphicspath{{2_chronisch_obstruktive_lungenerkrankung/figures/PNG/}{2_chronisch_obstruktive_lungenerkrankung/figures/PDF/}{2_chronisch_obstruktive_lungenerkrankung/figures/}}
\else
    \graphicspath{{2_chronisch_obstruktive_lungenerkrankung/figures/EPS/}{2_chronisch_obstruktive_lungenerkrankung/figures/}}
\fi

\chapter{Chronisch obstruktive Lungenerkrankung (COPD)}
\label{copd}
\setlength{\epigraphwidth}{6.0cm}
\epigraph{Learn extensively, inquire thoroughly, ponder prudently, distinguish clearly and practice devotedly.}{\emph{Liji$\cdot$Zhongyong}}


%\begin{figure}
% \centering
%  \includegraphics[width=0.48\textwidth]{task_classification}
%  \caption{Subtasks of automatic model learning, from \citet{fairfield2009phd}}
%  \label{figure_task_classification}
%\end{figure}

Die Chronisch obstruktive Lungenerkrankung, auf englisch Chronic Obstructive Pulmonary Disease (COPD) genannt, ist trotz seiner starken Verbreitung noch vielen Menschen nicht bekannt.

\section{Grundlagen} % section headings are printed smaller than chapter names
\label{grundlagen}

\subsection{Definition}
\label{definition}
Bei der COPD handelt es sich um eine chronische Lungenkrankheit mit progredienter Atemwegsverengung ; sie gilt als die häufigste Erkrankung der Atmungsorgane. COPD ist ein Sammelbegriff für die chronisch obstruktive Bronchitis und das Lungenemphysem, welche entweder einzeln oder gemeinsam auftreten (vgl. Checkliste XXL xx). Um das Krankheitsbild, Diagnostik und Therapie weltweit zu vereinheitlichen, wurde der o.g. englische Begriff eingeführt. Eine globale Initiatve, welche 2001 von der Weltgesundheitsorganisation (WHO) und den National Institutes of Health (NIH) gegründet wurde, hat zusätzlich dazu beigetragen, dass weltweit eine einheitliche Leitlinie, die s.g. GOLD-Leitlinie, zum Tragen kommt. Die Erkrankung wird hiernach in vier Schweregrade eingeteilt, die sich an dem gemessenen Ausatemvolumen orientieren. Einige Autoren, darunter Köhler/Schönhofer/Voshaar (vgl. 2010: 75), bewerten diese Leitlinie jedoch als einen Rückschritt, da diese das Krankheitsbild in Bezug auf seine Pathophysiologie zu sehr vereinfache.
Als ein wichtiges Diagnosemerkmal gilt die Atemnot, welche aus der Obstruktion der Bronchien resultiert. Diese wird durch drei Faktoren ausgelöst: 

\begin{enumerate}
\item Verkrampfung der Bronchialmuskulatur (Bronchospasmus)
\item Anschwellen der Schleimhaut in den Bronchien (Ödem)
\item Krankhaft erhöhte Schleimproduktion (Hyperkrinie) aufgrund einer dauerhaften Entzündung der Atemwege (chronische Bronchitis)
\end{enumerate}

Die Luftnot tritt im Anfangsstadium nur unter Belastung auf, später auch im Ruhezustand.
Eine chronisch obstruktive Lungenerkrankung liegt dann vor, wenn Husten und Auswurf bereits etwas länger als ein Jahr bestehen und somit als chronisch anzusehen sind als auch andere Krankheiten wie z.B. Bronchiektasen, Staubbelastung, cystische Fibrose, Asthma, Fremdkörper u.a. im Vorfeld ausgeschlossen werden konnten (vgl. Köhler/Schönhofer/Voshaar 2010: 71).

\subsection{Epidemiologie}
\label{epidemiologie}
Das Auftreten der COPD nimmt mit dem Alter zu, dabei kommt es ab dem 50. Lebensjahr zu einem rapiden Anstieg der Prävalenz. Im siebten Dezennium ist die Spitze des Auftretens mit etwa 10\% bei Männern und ca. 5\% bei Frauen erreicht (vgl. Lorenz 2009: 153). Nach Köhler/Schönhofer/Voshaar sei Vorsicht geboten bei dem Einbezug von Statistiken in diesem Bereich, da die verfügbaren Daten zur Prävalenz der COPD [XXX] sehr von dem untersuchten Kollektiv bzw. den Altersgruppen abhängen(2010: 72). Insgesamt gehen sie jedoch von folgenden Zahlen aus: Etwa 10\% der deutschen Bürger ab dem 40. Lebensjahr sei davon betroffen, wobei ca. 10\% dieser Patientenkohorte einen höheren Schweregrad aufweisen. Es wird davon ausgegangen, dass den Hausärzten nur etwa bei der Hälfte der Patienten die Erkrankung bekannt ist.
In Bezug auf die Mortalität gilt bei ca. 3,5\% aller Todesfälle COPD als die Haupttodesursache. Allerdings ist sie bei ca. 4,5\% der Todesfälle mitverursachend. Es wird davon ausgegangen, dass COPD auf der Liste der häufigsten Todesursachen in den nächsten 6 Jahren weltweit von dem 4. auf den 3. Platz aufsteigen wird.

\subsection{Aetiologie}
\label{aetiologie}

\subsection{Pathogenese und Pathophysiologie}
\label{pathogenese}

\subsection{Schweregrade}
\label{schweregrade}
In den GOLD-Leitlinien, welche bereits im Kapitel \ref{definition} erwähnt wurden, wird zwischen 4 Stadien unterschieden. Für die Einteilung kommen 2 Werte der Lungenfunktionsprüfung (siehe \ref{diagnostik}) zum Tragen. Der erste Wert, FEV1 (Einsekundenkapazität) gibt Auskunft darüber, wieviel Luft eine Person innerhalb einer Minute forciert ausatmen kann. Für eine Einstufung wird der FEV1-Wert eines Patienten mit Soll- bzw. Normalwerten verglichen, welche wiederum abhängig von Geschlecht, Alter und Körpergröße des Patienten sind.

\section{Symptomatik}
\label{symptomatik}

\section{Diagnostik}
\label{diagnostik}

\section{Therapie}
\label{therapie}

\subsection{Medikamentoese Therapien}
\label{medikamtentoese therapien}

\subsection{nicht-medikamentoese Therapien}
\label{nicht-medikamentoese therapien}

\section{Komorbiditaeten}
\label{komorbiditaeten}


\section{Zusammenfassende Betrachtung}
\label{zusammenfassende betrachtung}

% ---------------------------------------------------------------------------
% ----------------------- end of thesis sub-document ------------------------
\newpage\thispagestyle{empty}