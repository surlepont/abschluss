\ifpdf
\graphicspath{{2_chronisch_obstruktive_lungenerkrankung/figures/PNG/}{2_chronisch_obstruktive_lungenerkrankung/figures/PDF/}{2_chronisch_obstruktive_lungenerkrankung/figures/}}
\else
    \graphicspath{{2_chronisch_obstruktive_lungenerkrankung/figures/EPS/}{2_chronisch_obstruktive_lungenerkrankung/figures/}}
\fi

\chapter{Chronisch obstruktive Lungenerkrankung (COPD)}
\label{copd}
\setlength{\epigraphwidth}{6.0cm}
\epigraph{Learn extensively, inquire thoroughly, ponder prudently, distinguish clearly and practice devotedly.}{\emph{Liji$\cdot$Zhongyong}}


%\begin{figure}
% \centering
%  \includegraphics[width=0.48\textwidth]{task_classification}
%  \caption{Subtasks of automatic model learning, from \citet{fairfield2009phd}}
%  \label{figure_task_classification}
%\end{figure}


\section{Grundlagen} % section headings are printed smaller than chapter names
\label{grundlagen}

Die Chronisch obstruktive Lungenerkrankung, auf englisch Chronic Obstructive Pulmonary Disease (COPD) genannt, ist trotz seiner starken Verbreitung noch vielen Menschen nicht bekannt.
Bei der COPD handelt es sich um eine chronische Lungenkrankheit mit progredienter Atemwegsverengung ; sie gilt als die häufigste Erkrankung der Atmungsorgane. COPD ist ein Sammelbegriff für die chronisch obstruktive Bronchitis und das Lungenemphysem, welche entweder einzeln oder gemeinsam auftreten \autocite[vgl.][153]{checkliste}.  Die COPD resultiert aus einer langfristigen Entzündung der Atemwege, welche durch ständige Belastung mit in erster Linie Zigarettenrauch, aber auch Umweltfaktoren, Staubpartikel und giftigen Dämpfen entsteht. Ätiologisch wird davon ausgegangen, dass 80-90\% der COPD-Patienten die Erkrankung aufgrund von Nikotinabusus entwickelt haben; das sind in etwa 15\% der Zigarettenraucher\autocite[vgl.][154]{lorenz2009}. 

Als ein wichtiges Diagnosemerkmal gilt die Atemnot (Dispnoe), welche aus der Obstruktion der Bronchien resultiert. Diese wird durch drei Faktoren ausgelöst: 

\begin{enumerate}
\item Verkrampfung der Bronchialmuskulatur (Bronchospasmus)
\item Anschwellen der Schleimhaut in den Bronchien (Ödem)
\item Krankhaft erhöhte Schleimproduktion (Hyperkrinie) aufgrund einer dauerhaften Entzündung der Atemwege (chronische Bronchitis)
\end{enumerate}

Die Luftnot tritt im Anfangsstadium nur unter Belastung auf, später auch im Ruhezustand \autocite[vgl.][6f.]{lorenz2009}. Die allgemeine Symptomatik umfasst morgendlichen Kopfschmerz, Gewichtsverlust, welcher auf verstärkte Atemarbeit und systemische Entzündungsreaktion zurückzuführen ist, zunehmender Leistungsabfall, Einflussstauung, Ödeme der unteren Extremitäten und weiter abnehmende Belastbarkeit aufgrund des s.g. Cor pulmonale (Lungenherz, s. weiter unten).
Eine chronisch obstruktive Lungenerkrankung liegt laut WHO-Definition dann vor, wenn Husten und Auswurf über wenigstens als 3 Monate in mindestens 2 aufeinander folgenden Jahren bestehen und somit als chronisch anzusehen sind als auch andere Krankheiten wie z.B. Bronchiektasen, Staubbelastung, cystische Fibrose, Asthma, Fremdkörper u.a. im Vorfeld ausgeschlossen werden konnten \autocite[vgl.][71]{koehler2010}. 

%todo Grafik der Lunge als auch der Lungenbläschen

Das Lungenemphysem ist das Resultat einer länger andauernden Entzündungsreaktion auf o.g. mögliche externe Partikel. Durch entzündliche Prozesse  werden die Zellwände in den Alveolen  (Lüngenbläschen)(siehe Abbildung todo) zerstört. Hierfür sind Proteasen zuständig, die bei Eindringen von schädlichen Stoffen in die Lunge durch das Immunsystem freigesetzt werden. So genannte Antiproteasen können jedoch vor der Zerstörung der Alveolarwände schützen. Diesewerden für normal körpereigen produziert; hierzu gehört u.a. das Alpha-1-Antitripsin. Einige Patienten leiden jedoch unter einem Alpha-1-Antripsinmangel, wodurch ein erhöhtes Risiko für die Ausbildung eines Lungenemphysems besteht.
Das bedeutet, dass sich mit weiterbestehender Entzündung die Anzahl der für den Sauerstoff-Kohlendioxid-Austausch notwendigen Alveolen verringert und sich die Lufträume in der Lunge vergrößern. Dadurch wird nach und nach die Lungenelastizität eingeschränkt, was eine Überdehnung der Lunge (Hyperinlation) mit Minderdurchblutung und einem irreversiblen Schwund an Lungengewebe nach sich zieht. Dadurch wird die Atemfunktion immer mehr eingeschränkt, jedoch nicht nur durch die degenerativen Prozesse in einem Lungenlappen, sondern auch durch funktionelle Beeinträchtigung anderer umliegender gesunder Lungenlappen aufgrund der partiellen Überblähung. Auch andere Organe können möglicherweise beschädigt werden. Denn bei Nichtbehandlung des Emphysems kommt es zu einer erhöhten Pumpleistung des Herzmuskels und so über längere Zeit zu einer Schädigung desgleichen, da dieser nun mehr Blut transportieren muss, um die Organe mit ausreichendem Sauerstoff zu versorgen. Daher ist die Todesursache bei einem Lungenemphysem in den meisten Fällen ein Herzversagen und nicht, wie evtl. zuvor vermutet, Erstickung.

Bei Fortschreiten der COPD kommt es immer wieder zu Exazerbationen, das heißt zu akuten Verschlechterungen der alltäglichen Krankheitssymptome, was zu Symptomen wie Atemnot, Husten, vermehrter Sputummenge oder Fieber führen kann. Sie resultieren zu 60-70\% aus einer Infektion der Lunge oder nichtinfekiösen Ursachen wie akute Luftverschmutzung oder Verschlechterung von Begleiterkrankungen und sollten daher zeitnah behandelt werden.

Das Auftreten der COPD nimmt mit dem Alter zu, dabei kommt es ab dem 50. Lebensjahr zu einem rapiden Anstieg der Prävalenz. Im siebten Dezennium ist die Spitze des Auftretens mit etwa 10\% bei Männern und ca. 5\% bei Frauen erreicht\autocite[vgl.][153]{lorenz2009}. Nach Köhler/Schönhofer/Voshaar sei Vorsicht geboten bei dem Einbezug von Statistiken in diesem Bereich, da die verfügbaren Daten zur Prävalenz der COPD sehr von dem untersuchten Kollektiv bzw. den Altersgruppen abhängen(2010: 72). Insgesamt gehen sie jedoch von folgenden Zahlen aus: Etwa 10\% der deutschen Bürger ab dem 40. Lebensjahr sei davon betroffen, wobei ca. 10\% dieser Patientenkohorte einen höheren Schweregrad aufweisen. Es wird davon ausgegangen, dass den Hausärzten nur etwa bei der Hälfte der Patienten die Erkrankung bekannt ist.
In Bezug auf die Mortalität gilt bei ca. 3,5\% aller Todesfälle COPD als die Haupttodesursache. Allerdings ist sie bei ca. 4,5\% der Todesfälle mitverursachend. Es wird davon ausgegangen, dass COPD auf der Liste der häufigsten Todesursachen in den nächsten 6 Jahren weltweit von dem 4. auf den 3. Platz aufsteigen wird.
Laut der COPD-Leitlinie besteht eine hohe sozioökonomische Belastung durch die steigende Zahl an COPD-Fällen. In der Hochrechnung von Krankenhausstatistiken seit 1996 wurden für obstruktive Atemwegserkrankungen 2,7 Mio. Krankenhaustage erfasst, einen Großteil macht hierbei die Behandlung chronischer Bronchitis und ihrer Folgen aus. Auch die von der AOK hochgerechneten jährlichen Krankheitstage in Höhe von 25 Mio. aufgrund der chronischen Bronchitis sind immens. Sie entsprechen volkswirtschaftlichen Gesamtkosten von etwa 5,93 Mrd. \autocite[vgl.][e4]{vogelmeier2007}.

Um das Krankheitsbild, Diagnostik und Therapie weltweit zu vereinheitlichen, wurde der o.g. englische Begriff eingeführt. Eine globale Initiatve, welche 2001 von der Weltgesundheitsorganisation (WHO) und den National Institutes of Health (NIH) gegründet wurde, hat zusätzlich dazu beigetragen, dass weltweit eine einheitliche Leitlinie, die s.g. GOLD-Leitlinie, zum Tragen kommt. Die Erkrankung wird hiernach in vier Schweregrade eingeteilt, die sich an dem gemessenen Ausatemvolumen orientieren. Einige Autoren, darunter Köhler/Schönhofer/Voshaar \autocite[vgl.][75]{koehler2010}, bewerten diese Leitlinie jedoch als einen Rückschritt, da diese das Krankheitsbild in Bezug auf seine Pathophysiologie zu sehr vereinfache.

Diese Einteilung wird jedoch in der Praxis nach wie vor vorgenommen, weshalb sie an dieser Stelle dargestell werden soll.

In den GOLD-Leitlinien wird zwischen 4 Stadien unterschieden. Für die Einteilung kommen 2 Werte der Lungenfunktionsprüfung (siehe \ref{diagnostik}) zum Tragen. Der erste Wert, das forcierte exspiratorische Volumen (FEV1) gibt Auskunft darüber, wieviel Luft eine Person innerhalb einer Minute forciert ausatmen kann. Für eine Einstufung wird der FEV1-Wert eines Patienten mit Soll- bzw. Normalwerten verglichen, welche wiederum abhängig von Geschlecht, Alter und Körpergröße des Patienten sind. Dieser Wert wird in der Regel als Indikator für die Schwere der Erkrankung. In diesem Zusammenhang ist auch das Verhältnis zwischen der inspiratorischen Vitalkapazität (Einatemvolumen) und dem FEV1 für die Diagnosestellung wichtig (siehe Tabelle todo). Bei der leichtgradigen COPD (Schweregrad I) besteht die Atemwegsobstruktion ohne eine signifikante FEV1-Verminderung. Patienten in diesem Stadium berichten über chronischen Husten und Auswurf, bemerken jedoch meist noch keine Einschränkung ihrer Lungenfunktion. Beim Schweregrad II besteht neben der Atemwegsobstruktion bereits eine geringe FEV1- Verminderung und die kranksheitsspezifischen Symptome, insbesondere Dyspnoe unter Belastung, nehmen zu. Die schwere COPD (Schweregrad III) ist charakterisiert durch eine höhergradige FEV1-Verminderung, das heißt FEV1-Werte zwischen 30\% und 50\% des Soll. Es besteht jedoch nur eine geringe oder keine Korrelation zwischen dem Ausmaß der Dyspnoe und dem Schweregrad der Lungenfunktionseinschränkung. Für den Schweregrad IV gilt ein FEV1-Wert von todo kleiner=30\% Soll als ausschlaggebend. Bei einer gleichzeitigen respiratorischen Insuffizienz darf für die Einteilung in den Schweregrad IV der FEV1-Wert <50\% soll betragen \autocite[vgl.][e8]{vogelmeier2007}..

todo Grafik Schweregrade

Eine neuere, multidimensionale Schweregradbeurteilung stellt der s.g. BODE-Index dar (siehe Tabelle todo). Dieser bezieht in seine Bewertung den Body-mass-index (B), die FEV1-Einschränkung (O, Obstruction), das Dyspnoeempfinden (D) und die Belastbarkeit (E, exercise capacity) mit ein. Dabei wird eine Aussage über das Dyspnoeempfinden mittels des leicht abgeänderten Medical Research Council (MRC)-Scores vorgenommen, der folgendermaßen eingeteilt ist: "0 = keine Atemnot, 1 = Atemnot bei schwerer Belastung, 2 = Atemnot bei leichter Belastung, 3 = zu atemlos, das Haus zu verlassen und atemlos beim An- und Ausziehen" \autocite[186f.]{welte2007}. Die Belastbarkeit wird durch den 6-Minuten-Gehtest gemessen, welcher sich nach der zurückgelegten Strecke in m orientiert. 

todo Grafik BODE

\section{Diagnostik}
\label{diagnostik}
Um eine Verdachtsdiagnose stellen zu können, wird zu Beginn eine umfangreiche \emph{Anamnese} durchgeführt. Diese umfasst folgende, auf COPD verweisende Kriterien: Alter, Familienanamnes, Husten, Auswurf, Atemnot unter Belastung, Rauchgewohnheit und/oder inhalative Belastung am Arbeitsplatz, Anzahl der Exazerbationen pro Jahr, gegenwärtige Medikation, Beeinträchtigung im Alltag, Sozialanamnese, Störungen der Atmung im Schlaf, mögliche Komorbiditäten (s. weiter unten) und Gewichtsverlust. 

Folgende \emph{körperliche Untersuchungsbefunde} geben ebenfalls Hinweis auf eine mögliche COPD, wobei bei einer leichtgradigen Ausprägung der Erkrankung diese Befunde unauffällig sein können: bei einer mittelgradigen COPD deuten verlängertes Exspirium, Giemen, Pfeifen und Brummen auf eine Obstruktion der Atemwege, ein tief stehendes, wenig verschiebbares Zwerchfell und hypersonorem Klopfschall auf eine Lungenüberblähung hin. Für eine schwere Form der Erkrankung sind in der COPD-Leitlinie folgende Untersuchungsbefunde kennzeichnend:
\begin{itemize}
\item "Zeichen der chronischen Lungenüberblähung mit abgeschwächtem Atemgeräusch, leisen Herztönen, Fassthorax und inspiratorischen Einziehungen im Bereich der Flanken,
\item pfeifende Atemgeräusche, insbesondere bei forcierter Exspiration,
\item Zeichen der Sekretansammlung im Anhusteversuch,
\item zentrale Zyanose,
\item Konzentrationsschwäche und verminderte Vigilanz,
\item Gewichtsverlust,
\item periphere Ödeme,
\item indirekte Zeichen der pulmonalen Hypertonie mit präkordialen Pulsationen, betontem Pulmonalklappenschlusston, einer Tricuspidalklappeninsuffizienz mit einem Systolikum über
dem 3. bzw. 4. ICR rechts parasternal." \autocite[e6]{vogelmeier2007}
\end{itemize}
Es müssen für eine COPD-Diagnosestellung jedoch nicht alle genannten Symptome vorliegen.

Einen sehr wichtigen Teil der Diagnostik stellen \emph{Lungenfunktionstestungen} dar, weil sie zum Einen für die Einteilung der Schweregrade gebraucht wird und andererseits Aussagen über eine nicht vollständig reversible Atemwegsobstruktion durch Nicht-Ansprechen auf die Gabe von Bronchodilatatoren und/oder Glukokortikoiden (siehe auch \label{medikamentoese therapien} treffen kann, was ein klares Indiz für eine ausgebildete COPD darstellt. Dieser Reversibilitätstest ist insbesondere für die Differenzialdiagnose zwischen Asthma und COPD entscheidend, welche für das Management der COPD sehr wichtig ist. Die charakteristischen Merkmale beider Erkrankungen sind in Tabelle todo gegenübergestellt.

Die gängigsten Lungenfunktionstests stellen die Spirometrie und Ganzkörperplethysmographie dar. Bei der Spirometrie wird durch Ein- und Ausatmen in ein hierfür spezialisiertes Gerät gemessen, wie viel Luft durch die Lunge aufgenommen und wie schnell diese gefüllt und wieder geleert werden kann. Hier wird der oben mehrmals erwähnte FEV1-Wert als auch die forcierte Vitalkapazität (FVC) ermittelt. Je niedriger der FEV1-Wert ausfällt desto schlechter ist die Lungenkapazität. Bei der Ganzkörper-, oder auch Bodyplethysmographie genannt, wird hingegen gemessen, wieviel Luft in der Lunge nach der maximalen Ausatmung in der Lunge verbleibt. Das Ergebnis bildet sich in der s.g. funktionellen Residualkapazität (FRC) ab \autocite[vgl.][e6f.]{vogelmeier2007}. Bei einem Lungenemphysem wird das Residualvolumen größer sein als bei vergleichbaren gesunden Menschen. In bestimmten Fällen, wie bei Patienten mit einer Diskrepanz zwischen Dyspnoe und Einschränkung der FEV1 oder bei Patienten der Schweregrade III und IV, können zur weiteren Abklärung noch andere Messverfahren eingesetzt werden.

Ein weiteres Verfahren stellt die \emph{arterielle Blutgasanalyse} dar, welche zur Abklärung einer Gasaustauschstörung, auch respiratorische (Partial-)Insuffizienz genannt, sowie zur therapeutischen Abschätzung der Indikation für eine Sauerstofftherapie dient. Die Bestimmung der Blutgase geschieht über Analyse von hyperämisiertem Kapillarblut, welches i.d.R. dem Ohrläppchen entnommen wird. "Eine respiratorische Partialinsuffizienz wird bei Sauerstoff-Partialdruck (PaO2)-Werten < 8,0 kPa (60 mmHg) diagnostiziert, eine Hyperkapnie bei einem CO2-Partialdruck (PaCO2) > 6,0 kPa [45 mm Hg]. Bei Patienten mit schwerer COPD kann eine respiratorische Globalinsuffizienz mit arterieller Hypoxämie und Hyperkapnie angetroffen werden" \autocite[190]{welte2007}.

Wie bereits oben erwähnt, gehören auch \emph{Belastungstests} zur COPD-Diagnostik. Diese sollen Aufschluss über die verschiedenen Ursachen der Belastungsdyspnoe und die Therapieeffekte von Medikamenten und Orientierung zur Erstellung eines individuell angepassten Trainingsprogramms im ambulanten als auch rehabilitativen Bereich geben \autocite[vgl.][e7]{vogelmeier2007}. 

\emph{Bildgebende Verfahren} sind für die Diagnostizierung eines Lungenemphysems sowie für die Differenzialdiagnostik (insbesondere zum Ausschluss eines Bronchialkarzinoms oder kardialer Erkrankungen) von großer Bedeutung. Hierfür werden i.d.R. Röntgenaufnahmen und Computertomografie (High Resolution, HR-CT) der Thoraxorgane eingesetzt. Die HR-CT wird insbesondere vor einer Resektion von Emphysem-Bullae (Emphysemblasen) oder einer Lungenvolumenreduktion (siehe \label{nicht-medikamentoese therapien} durchgeführt \autocite[191]{welte2007}.

Bei Patienten mit häufigen Exazerbationen (>3 pro Jahr), Therapieversagern und/oder bei besonders schweren Erkrankungen, die auf eine multiresistente bakterielle Besiedelung der Lunge hindeuten, wird die \emph{Sputumdiagnostik} empfohlen. Patienten sind in der Regel in der Bewertung ihres eigenen Sputums, auch bekannt als Auswurf oder Sekret, geschult, um Veränderungen ihres Krankheitszustandes frühzeitig erkennen zu können. 

Zudem werden aufgrund des Zusammenhangs der COPD mit kardialer Belastung auch Echokardiographie und Elektrokardiogramm eingesetzt.

Da es sich bei der COPD um eine progrediente Erkrankung mit fließenden Übergängen handelt, bedarf es einer aufmerksamen und regelmäßigen fachärztlichen Versorgung, welche die Funktionsparameter und klinischen Symptome mind. einmal jährlich bzw. bei Verschlechterung der Krankheitssymptome überprüft \autocite[vgl.][e8ff.]{vogelmeier2007}.

Die zuvor beschriebenen Inhalte der Diagnostik bei COPD sind in Tabelle todo nochmals als nachvollziehbarer Algorithmus dargestellt.

\section{Behandlungsmethoden}
\label{behandlungsmethoden}
Die Behandlung bei COPD schließt neben Raucherentwöhnung, medikamentöser Therapie, Patientenschulung, Physiotherapie, körperlichem Training, Ernährungsberatung sowie apparativer Therapiemöglichkeiten bei schwererem Lungenemphysem auch chirurgische Maßnahmen ein; traditionell läge der Schwerpunkt jedoch auf der medikamentösen Therapie, so Lang \autocite[vgl.][287]{lang2007}. Die Therapieziele umfassen in erster Linie die Rückbildung von Dyspnoe, Husten und Auswurf, Verbesserung der Belastbarkeit, längere Lebenserwartung und die Senkung der Exazerbationsfrequenz. Atemphysiologisch wird ein Anstieg des FEV1- und p2O2 todo -Wertes angestrebt als auch ein Abfall von Atemwegswiderstand, thorakalem Gasvolumen, Residualvolumen und paCO2 todo \author[vgl.][158]{lorenz2009}
Die o.g. verschiedenen Behandlungszweige werden in den beiden folgenden Kapiteln kurz dargestellt.

\subsection{Medikamentoese Therapien}
\label{medikamentoese therapien}
Aufgrund häufiger Komorbiditäten bei COPD und mehreren Symptomen dieser komplexen Erkrankung bedarf es einer individuell angepassten pharmazeutischen Behandlung der Patienten, welche pneumologisch gut abgeklärt sein sollte. Die medikamentöse Behandlung hat keinen Einfluss auf den progredienten Verlauf der Erkrankung, welche mit Einschränkungen der Lungenfunktion einhergeht. Sie könne jedoch zur Linderung der Beschwerden, einer Verbesserung der körperlichen Leistungsfähigkeit, der Lebensqualität und zur Reduktion von Exazerbationen beitragen \autocite[vgl.][249]{gillissen2007}. In der Fachliteratur wird immer wieder darauf verwiesen, dass die Pharmakotherapie bei Rauchern mit COPD immer flankiert sein sollte durch Raucherentwöhnungsprogramme (siehe \label{nicht-medikamentoese therapien}), um die Langzeitprognose zu verbessern. Die psychosoziale Begleitung in Verbindung mit einer Nikotinersatztherapie und einer angemessenen Nachsorge sowie Rückfallintervention gilt als besonders erfolgsversprechend. Die Nikotinersatztherapie beinhaltet die Gabe eines Nikotinersatzes in Form von Kaugummi, Pastillen, Pflaster u.a., welche an den Körper Nikotin langsamer und in sehr geringer Menge abgiebt als beim aktiven Rauchen. Hierdurch sollen die Entzugserscheinungen beim Rauchen reduziert werden. Darüber hinaus kann die Vergabe des Antidepressivums Bupropion zusätzlich die Entwöhnungsrate steigern \autocite[vgl.][e12]{vogelmeier2007}.

Bei der Pharmakotherapie bildet die Gruppe der Bronchodilatatoren (β2-Sympathomimetika, Anticholinergika, Theophyllin) die Basismedikation. Medikamente dieser Wirkstoffgruppe entspannen die Muskeln in den verengten Atemwegen und verbessern so die Luftzufuhr, jedoch setzen sie an unterschiedlichen Stellen an. Sie werden unterteilt in kurz- und langwirksame Bronchodilatatoren entsprechend der Dauer der Wirksamkeit und des Eintritts der gewünschten Wirkung. Als Dauermedikation der COPD werden langwirksame Bronchodilatatoren wie β2-Sympathomimetika (Formoterol, Salmeterol) und Anticholinergika (Triotropiumbromid) eingesetzt. Die Entscheidung, welches Medikament für den jeweiligen Patienten am geeignetsten ist, hängt von dessen individuellem Ansprechen auf den Wirkstoff in Bezug auf erwünschte Effekte ab \autocite[vgl.][e13]{vogelmeier2007}. β2-Sympathomimetika, oder auch -Agonisten, entspannen die Muskeln in den Bronchiolen, wohingegen Anticholinergika hauptsächlich auf die großen Atemwege wirken. Aufgrund vieler Nebenwirkungen und Interaktionen aber auch wegen der geringeren bronchodilatativen Wirkung im Vergleich zu β2-Sympathomimetika wird Theophyllin möglichst nur eingesetzt, wenn die anderen Medikamente nicht greifen. 



\subsection{nicht-medikamentoese Therapien}
\label{nicht-medikamentoese therapien}
Als vorrangiges Ziel der Prävention gilt die Reduktion inhalativer Schadstoffe im näheren Umfeld des Patienten. Hierbei steht der Verzicht auf Nikotinkonsum an oberster Stelle, weshalb ein wichtiger Bereich der nicht-medikamentösen Therapie \emph{Raucherentwöhnungsprogramme} darstellen. Durch einen Rauchstopp kann ein schnelleres Fortschreiten der Erkrankung verhindert und eine bessere Voraussetzung für eine erfolgreiche Behandlung geschaffen werden. Daneben kann so die Prävalenz von Exazerbationen und das Risiko für Komorbiditäten gesenkt werden.

In der \emph{physiotherapeutischen Atemtherapie} erlernen die Patienten einen bewussten Umgang mit ihrer Atmung mithilfe von Atemerleichternden Stellungen sowie Atemtechniken. Eine sehr wichtige Atemtechnik zur Verringerung der akuten Atemnot stellt hierbei die „Lippenbremse“ dar. Hierbei wird die Luft durch die besondere enge Stellung der Lippen sehr dosiert ausgeatmet und wirkt durch den aufgebauten Druck bronchienerweiternd. Diese Atemtechnik wird möglichst mit atemerleichternden Stellungen kombiniert (Kutschersitz, „Hängebauchschwein“, Torwartstellung, Wandstütze u.a.), welche den Thorax vom Gewicht des Schultergürtels entlasten und zudem das „Längen-Spannungsverhältnis der Atemhilfsmuskulatur und des Zwerchfells“ verbessern \autocite[vgl.][291]{lang2007} todo. Ein weiterer wichtiger Bereich der Atemtherapie bildet die Koppelung von Atmung und Bewegung. Hierbei wird die eigene Atmung von körperlichen Bewegungen, wie z.B. im Liegen beim Einatmen den Arm über die Seite nach oben und beim Ausatmen wieder zurückführen, begleitet. Hierdurch werden Belastungssituationen in einem geschützten Rahmen simuliert und erprobt.

Besonders wichtig für die Prävention von Exazerbationen ist die s.g. Sekretdrainage, ein Bereich der \emph{physikalischen Therapie}. Hier gilt es, durch bestimmte Körperlagerungen und Handgriffe, welche die Atembewegung unterstützen oder einer erhöhten Widerstand erzeugen, sowie durch Vibrationen das Sekret zu mobilisieren und so das Abhusten zu erleichtern. 

Ergotherapie

Patientenverhaltenstraining

Entspannungsverfahren

Hilfsmittel

Ein weiterer wichtiger Bereich stellt hier die \emph{pneumologische Rehabilitation} dar. 

\section{Komorbiditaeten}
\label{komorbiditaeten}
Das Thema „Begleiterkrankungen bei COPD“ ist bereits zuvor an einigen Stellen angeklungen. Obgleich Leitlinien den Einfluss fachübergreifender Komorbiditäten oftmals außer Acht lassen, sind diese für die gesundheitsbezogene Lebensqualität laut König wichtiger als das FEV1, demografische Faktoren oder Atemwegssymptome. Auch der Einfluss von Komorbiditäten auf die Mortalität stellt einen signifikanten Indikator dar \autocite[vgl.][395]{koenig2007}.
Daher soll diesem wichtigen Aspekt an dieser Stelle Rechnung getragen werden und anschließend auf körperliche und psychische Begleiterkrankungen näher eingegangen werden. Letzteres wird jedoch aufgrund der hohen Relevanz für das in dieser Arbeit entwickelte musiktherapeutische Konzept schwerpunktmäßig behandelt.


\subsection{Körperliche Komorbiditäten}
Wie bereits zu Beginn des \label{COPD} erläutert, entwickeln viele COPD-Patienten im Krankheitsverlauf körperliche Begleiterkrankungen. Sehr häufig kommt es zu Herz-Kreislauf-Erkrankungen, die aus einem unzureichenden Gasaustausch resultieren. Die Herzinsuffizienz tritt so beispielsweise bei ca. 30\% der COPD-Patienten auf und für jeden 2. bestehe laut Stiefelhagen eine arterielle Hypertonie. 25\% der COPD-Betroffene leiden zudem unter einer hämodynamisch wirksamen KHK, was vermutlich dem Nikotinabusus geschuldet ist \autocite[vgl.][37]{stiefelhagen2013}. Todo

Eine weitere häufige Komorbidität stellt Osteoporose dar. Stiefelhagen zufolge habe dies unterschiedliche Gründe: „Immobilität, Alter, Gewichtsverlust, Steroid-Medikation, die COPD-typische systemische Entzündung und auch Nikotinabusus“ \autocite[37]{stiefelhagen2013}. An anderen Stellen wird diese Begleiterkrankung oftmals als Nebenwirkung der medikamentösen Behandlung hervorgehoben. 

Die gleichzeitige Prävalenz von COPD und Typ-2-Diabetes korreliert ebenfalls in vielen Fällen. Dies wird auf die systemische Inflammation als ein „pathogenetisches Bindeglied“ ausgegangen, da sowohl die obstruktive Lungenerkrankung als auch der Diabetes mellitus Entzündungsprozesse beeinflussen.

Nicht zuletzt tritt im Zusammenhang mit der COPD auch immer wieder Lungenkrebs auf, welcher ebenfalls aus einem langjährigen Zigarettenkonsum resultiert \autocite[vgl.][38]{stiefelhagen2013}.


\subsection{Psychische Komorbiditäten}
\label{psychische_komorbiditaet}
Einen sehr wichtigen, vielfach nicht berücksichtigten Bereich bei COPD-Patienten stellen psychische Komorbiditäten dar.  Beinahe die Hälfte dieser Patientengruppe leidet zusätzlich zu ihrer körperlichen Erkrankung unter Angst- und Panikstörungen und/oder Depressionen; es wird zudem von einer hohen Dunkelziffer ausgegangen. Nach Steinkamp et. al kommen als Ursache für psychische Störungen bei COPD unterschiedliche Faktoren in Frage. Diese umfassen: Nikotinabhängigkeit, eigenständige psychiatrische Krankheit, Hypoxämie, Hyperkapnie, Atemnot, Schlafstörungen, Einschränkung der Belastbarkeit und Mobilität und Vereinsamung. Literaturangabe: COPD und Psyche 
 
In wie weit jedoch diese Psychopathologien als Folge der COPD gesehen können oder relevante Koinzidenzen darstellen, ist noch ungeklärt. 

In den meisten Fällen bezieht man sich in der Literatur auf ein Krankheitsmodell, welches in der GOLD-Leitlinie aber auch in der Leitlinie der Atemwesliga zu finden ist. Hierin wird von einem Teufelskreis, Circulus virtuosus, ausgegangen, wonach die körperliche Leistungseinschränkung aufgrund vorherrschender Atemnot im Verlauf der Erkrankung zum Rückzug aus alltäglichen Abläufen und so immer mehr zur Immobilität und sozialen Isolation führt. Dadurch werden jedoch auch Angst und Depression immer mehr verstärkt. Todo Tabelle 

Die Angaben zur Prävalenz von Angst und Depression bei COPD variieren sehr. „Generalisierte Angststörungen werden in einer Häufigkeit von 2-16\%, Panikstörungen von 8-67\%, depressive Symptome und Depressionen zwischen 11 und 80\% sowie Angstsymptome in einem Bereich von 10-75\% angegeben“  \autocite[34]{kenn2011} todo Unterschätzte Komorbidität Literatur.

An diesen Zahlen lässt sich erkennen, dass exakte Ergebnisse in Bezug auf die Prävalenz noch fehlen. Kenn und Kühl gehen davon aus, dass methodisch verschiedene Diagnoseansätze hierfür verantwortlich sind, wobei sie hierbei Ergebnisse aus Interviews, die sich an den Kriterien für psychische Störungen z.B. nach der ICD 10 orientieren, und Fragebögen gegenüberstellen. Generell erschwerten wohl neben den unterschiedlichen Erhebungsinstrumenten auch die Heterogenität der untersuchten Patientenkohorte mit verschiedenen Schweregraden die Interpretation der Ergebnisse \autocite[vgl.][35]{kenn2011}.
Zudem geben die Studienergebnisse keine Auskunft darüber, in wie weit bei Patienten, die eine COPD aufgrund eines längeren Nikotinabusus‘ entwickelt haben, bereits eine größere psychische Vulnerabilität und somit ein erhöhtes Risiko für die Ausbildung einer Depression oder Angst-/Panikstörung besteht. 

Ein evtl. wichtiges Konzept für den Krankheitsverlauf der COPD in seinen unterschiedlichen Facetten scheint die s.g. „Fear Avoidance“ zu sein. Viele Patienten berichten von Angst vor auftretender Atemnot. Fear Avoidance- Konzept meint „die Angst vor der Verstärkung eines Krankheitssymptoms bzw. Verschlechterung des Verlaufs und daraus folgende Aktivitätsvermeidung“ \autocite{stenzel20013} todo Literatur Internetartikel!. Dieses Konzept wurde jedoch bisher für COPD noch kaum diskutiert und erforscht. Eine neuere Studie zeigte jedoch, dass Fear Avoidance tatsächlich als Mediator des Zusammenhangs zwischen COPD-Status und Lebensqualität bzw. Gesundheitsstatus gesehen werden kann. Die Autoren plädieren daher dafür, dass im Rahmen der pneumologischen Rehabilitation auch psychotherapeutische Interventionen implementiert werden sollten, welche diesen Aspekt aufgreifen \autocite{stenzel20013}. 

In dem später dargestellten musiktherapeutischen Konzept wird dieser Aspekt ebenfalls implizit aufgenommen werden. Durch achtsames Wahrnehmen eigener Grenzen und Möglichkeiten, aber auch durch übungszentriertes Arbeiten entwickeln die Patienten durch positive Erfahrungen im Zusammenhang mit Selbstregulation wieder mehr Selbstvertrauen und erleben sich als selbstwirksame Individuen, die sich nicht der auftretenden Atemnot ausgeliefert fühlen müssen.

In Studien wurde zudem ein signifikanter Zusammenhang zwischen einer erhöhten Mortalität und einer COPD mit begleitender depressiver oder Angstsymptomatik nachgewiesen. Aber auch in Bezug auf die Exazerbations- und Rehospitalisationsrate sowie die Leistungsfähigkeit und -bereitschaft von COPD-Patienten scheint hier die Ausbildung der genannten psychischen Erkrankung einen relevanten negativen Faktor darzustellen\autocite[vgl.]{kenn2011}.

Obgleich das zuvor Erläuterte heutzutage unter allgemeinmedizinischen und pneumologischen Fachärzten als bekannt anzunehmen ist, wird die Problematik in den Arzt-Patient-Gesprächen oftmals nicht thematisiert. Eine amerikanische Studie zeigte diese Diskrepanz zwischen der Prävalenz und Behandlungshäufigkeit psychischer Erkrankungen bei gleichzeitiger COPD. In einer Telefonumfrage von 1334 Patienten lagen bei 61\% psychische Auffälligkeiten, insbesondere Angstsymptome, vor. Lediglich 31\% dieser Patienten wurden diesbezüglich behandelt \autocite[vgl.][156]{fischer2007}.


\section{Zusammenfassende Betrachtung}
\label{zusammenfassende betrachtung}

% ---------------------------------------------------------------------------
% ----------------------- end of thesis sub-document ------------------------
\newpage\thispagestyle{empty}