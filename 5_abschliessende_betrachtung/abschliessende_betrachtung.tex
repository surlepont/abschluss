\chapter{Abschließende Betrachtung} % top level followed by section, subsection
Ausblick: Der nächste Schritt bestünde nun in der praktischen Umsetzung dieses Konzepts. Hierfür würde sich m.E. im Hamburger Raum besonders die "Atem-Reha" am Berliner Tor anbieten, da hier das Setting der oben mehrfach beschriebenen pneumologischen Rehabilitation gegeben wäre. Die Wirksamkeitsüberprüfung könnte sich jedoch als schwierig herausstellen, da die Patienten natürlich nicht von anderen wichtigen Therapien, wie z.B. Atemtherapie, ausgeschlossen werden können. Es könnte jedoch über einen längeren Zeitraum 
Schwierigkeit: unterschiedliche Schweregrade, Hintergründe, Erfahrungen mit dem Singen... es müssten also recht viele Daten erhoben werden. In einem ersten Schritt könnten jedoch sowohl an Patienten, die an dem Angebot teilnehmen als auch an nicht teilnehmende, jedoch ebenfalls die pneumologische Rehabilitation in Anspruch nehmende Fragebögen ausgegeben werden, welche sich dem Thema "Lebensqualität" und "Selbstwirksamkeit" zuwenden, um einen Teil der Zielsetzungen bereits zu überprüfen. Hierfür gibt es bereits standardisierte Fragebögen, wie sie beispielsweise von Gunter Kreutz und Steven Clift in den zuvor genannten Studien eingesetzt wurden.



Kurz vor Abgabe dieser Arbeit wurde ich in meinem Ansinnen, ein musiktherapeutisches Konzept für die Arbeit mit COPD-Patienten, sehr gestärkt. Das Beth-Israel-Hospital in New York, welches derzeit mit zu den führendsten Forschungsinstituten für Musikmedizin und Musiktherapie zählt, stellt seit Ende Mai 2014 eine Teilnehmerkohorte für die Durchführung einer Studie zur Abklärung der Effekte auf die physische Funktionalität und Lebensqualität von Erwachsenen mit COPD mithilfe von Musiktherapie zusammen ("The Effects of Music Therapy in the Treatment of Chronic Obstructive Pulmonary Disease" ). Dies zeigt meines Erachtens die Aktualität dieses Themas und lässt hoffen, dass Musiktherapie evtl. in naher Zukunft in die Standardtherapie bei COPD integriert wird.



\setlength{\epigraphwidth}{7.5cm}
\epigraph{Study the past, if you would divine the future.}{Confucius}

\ifpdf
    \graphicspath{{X/figures/PNG/}{X/figures/PDF/}{X/figures/}}
\else
    \graphicspath{{X/figures/EPS/}{X/figures/}}
\fi

\lettrine{T}{his} dissertation 

\newpage\thispagestyle{empty}
% ---------------------------------------------------------------------------
%: ----------------------- end of thesis sub-document ------------------------
% ---------------------------------------------------------------------------