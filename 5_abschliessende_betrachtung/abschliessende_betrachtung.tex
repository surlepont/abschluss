\chapter{Abschließende Betrachtung} % top level followed by section, subsection
Kurz vor Abgabe dieser Arbeit wurde ich in meinem Ansinnen, ein musiktherapeutisches Konzept für die Arbeit mit COPD-Patienten, sehr gestärkt. Das Beth-Israel-Hospital in New York, welches derzeit mit zu den führendsten Forschungsinstituten für Musikmedizin und Musiktherapie zählt, stellt seit Ende Mai 2014 eine Teilnehmerkohorte für die Durchführung einer Studie zur Abklärung der Effekte auf die physische Funktionalität und Lebensqualität von Erwachsenen mit COPD mithilfe von Musiktherapie zusammen ("The Effects of Music Therapy in the Treatment of Chronic Obstructive Pulmonary Disease" ). Dies zeigt meines Erachtens die Aktualität dieses Themas und lässt hoffen, dass Musiktherapie evtl. in naher Zukunft in die Standardtherapie bei COPD integriert wird.



\setlength{\epigraphwidth}{7.5cm}
\epigraph{Study the past, if you would divine the future.}{Confucius}

\ifpdf
    \graphicspath{{X/figures/PNG/}{X/figures/PDF/}{X/figures/}}
\else
    \graphicspath{{X/figures/EPS/}{X/figures/}}
\fi

\lettrine{T}{his} dissertation 

\newpage\thispagestyle{empty}
% ---------------------------------------------------------------------------
%: ----------------------- end of thesis sub-document ------------------------
% ---------------------------------------------------------------------------