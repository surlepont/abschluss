\chapter{Schlussbetrachtung und Ausblick} % top level followed by section, subsection
Für die erläuterten konzeptionellen Überlegungen im vorherigen Kapitel war es wichtig, sich im Vorherein ausgiebig mit den Themen COPD und musiktherapeutische Stimmarbeit auseinanderzusetzen, um ein grundlegendes Verständnis und Wissen für beide Bereiche entwickeln zu können. So wurde beispielsweise erst im Verlauf der Bearbeitung dieser Masterthesis immer deutlicher, dass der Suchtaspekt in den konzeptionellen Überlegungen aufgegriffen und mitgedacht werden muss. Inwieweit nun aber die bereits bestehende Suchtstruktur ursächlich mit der Ausbildung einer Depression oder Angststörung zusammenhängt oder aber tatsächlich die COPD-Erkrankung erst zu dieser Entwicklung führt, kann hier nicht beantwortet werden, stellt jedoch m.E. ein sehr interessantes Forschungsfeld dar. In Zusammenhang mit einer eventuell bestehenden Ich-Schwäche, wie sie im Suchtkontext vermutet wird, stellt das Medium "`Stimme"' meiner Meinung nach eine gute Möglichkeit dar, diesen Bereich zu stärken. Durch die Arbeit mit der Stimme kommen wir direkt in Kontakt mit unserem eigenen Gefühlsleben und unserer Körperwahrnehmung, so dass wir über das In-uns-hinein-spüren und -hören einen besseren Zugang zu diesen erreichen können. Über diese sensibilisierte Wahrnehmung wird es immer leichter, eigene Bedürfnisse zu erkennen und in einem nächsten Schritt sich für diese einzusetzen. Wie es sich anfühlt, eigenen Impulsen, Wünschen und Gefühlen zu folgen, kann in diesem geschützten therapeutischen Rahmen im Sinne eines Probehandelns ausgetestet werden. Durch das unmittelbar Körperliche in diesem musikalischen Handeln können die neuen Erfahrungen sowohl körperlich verankert als auch durch die anschließende verbale Reflexion kognitiv als Handlungsalternativen integriert werden.

Die Idee des Einsatzes musiktherapeutischer Stimmarbeit als Begleitbehandlung bei einer COPD-Erkrankung beruht auf der Überlegung, sich die ganzheitliche Wirkung der Stimmarbeit auf körperlicher und psychischer Ebene auch in diesem Bereich nutzbar zu machen (siehe u.a. Kapitel \ref{wirkung_des_singens}). Gerade in Bezug auf ein gestörtes "`Körperselbst"', wie es weiter oben bereits im COPD-Kontext erläutert wurde, könnte die Form der musiktherapeutischen Arbeit einen positiven Einfluss auf und Veränderungen für dieses bedeuten: Im stimmlichen Ausdruck liegt die Chance, sich selbst als Ganzes zu erleben und wahrzunehmen sowie einen positiven Körperbezug zu stärken. Darüber hinaus ist die Stärkung der Selbstwirksamkeit und des eigenen gesundheitsorientierten Handelns ein wesentlicher Gesichtspunkt für eine verbesserte Lebensqualität und Gesundheitszustand COPD-Betroffener. Dies kann m.E. jedoch nur gelingen, wenn die Gefühle der Ohnmacht und des Ausgeliefertseins (siehe Kapitel \ref{psychodynamische_ueberlegungen}) überwunden werden können und so das eigene Gestalten wieder möglich wird. Mit Hilfe des Stimmausdrucks ist es in besonderer, körpernaher Weise möglich, sich selbst sowie das eigene körperliche und seelische Befinden zu beeinflussen. 

Auch die soziale Funktion des Singens scheint mir im therapeutischen Kontext mit COPD-Betroffenen ein wichtiger Aspekt hinsichtlich der belegten Tendenz zu sozialer Isolation zu sein, denn mittels der "Brückenfunktion der menschlichen Stimme zwischen "`Innen"' (...) und "`Außen"' \autocite[283]{deckervoigt2000} können Teilnehmer sich im Kontakt mit anderen erleben und so ihr "`soziales Selbst"' stärken. Dieser Aspekt hat nicht nur Einfluss auf die psychische Verfassung, die Lebensqualität und das Wohlbefinden Betroffener, sondern auch auf die Mortalität. Wie eine Studie englischer Forscher um Andrew Steptoe im vergangenen Jahr zeigte, erhöht "`soziale Isolation"' die Mortalität \autocite[vgl.][]{pmid23530191}. So ist die Reduzierung sozialer Isolation zudem wichtig im Hinblick die Senkung des vorzeitigen Todesrisikos.

%Die Wirksamkeitsüberprüfung könnte sich jedoch als schwierig herausstellen, da die Patienten natürlich nicht von anderen wichtigen Therapien, wie z.B. Atemtherapie, ausgeschlossen werden können und somit . Es könnte jedoch über einen längeren Zeitraum 
%Schwierigkeit: unterschiedliche Schweregrade, Hintergründe, Erfahrungen mit dem Singen... es müssten also recht viele Daten erhoben werden. In einem ersten Schritt könnten jedoch sowohl an Teilnehmer dieses Gruppenangebots als auch an nicht teilnehmende, jedoch ebenfalls die pneumologische Rehabilitation in Anspruch nehmende Patienten Fragebögen ausgegeben werden, welche sich dem Thema "`Lebensqualität"' und "`Selbstwirksamkeit"' zuwenden, um einen Teil der Zielsetzungen bereits zu überprüfen. Hierfür gibt es bereits standardisierte Fragebögen, wie sie beispielsweise von Gunter Kreutz und Steven Clift in den zuvor genannten Studien eingesetzt wurden.

Wie bereits zuvor anklang, bestünde nun der nächste Schritt in der praktischen Umsetzung dieses Konzepts. Hierfür würde sich m.E. im Hamburger Raum besonders die "`Atem-Reha"' am Berliner Tor anbieten, da hier das Setting der oben mehrfach beschriebenen pneumologischen Rehabilitation gegeben ist. So können bereits die praktischen Erfahrungen zu einer Weiterentwicklung des Konzepts führen. Darüber hinaus bedarf es hierfür der weiteren Diskussion und des Erfahrungsaustauschs mit Kollegen, die ebenfalls in diesem Umfeld tätig sind, sowie schließlich der wissenschaftlichen Überprüfung des Konzepts in der Praxis hinsichtlich der beschriebenen Zielsetzungen. Wie auch Clift et al. (siehe Kapitel \ref{copd_in_der_singforschung}) denke ich, dass in einem ersten Schritt überhaupt getestet werden sollte, ob eine größere Untersuchung hinsichtlich der erzielten Effekte überhaupt sinnvoll erscheint. Für die Überprüfung der Teilziele "`Steigerung der Lebensqualität und Selbstwirksamkeit"' könnten bereits entwickelte, standardisierte Fragebögen eingesetzt werden, welche schon von Clift et al. und Kreutz eingesetzt wurden.

Kurz vor Abgabe dieser Arbeit wurde ich in meinem Ansinnen, ein musiktherapeutisches Konzept für die Arbeit mit COPD-Patienten zu entwickeln, sehr gestärkt. Das Beth-Israel-Hospital in New York, welches derzeit mit zu den führenden Forschungsinstituten für Musikmedizin und Musiktherapie zählt, stellt seit Ende Mai 2014 eine Teilnehmerkohorte für die Durchführung einer Studie zur Abklärung der Effekte auf die physische Funktionalität und Lebensqualität von Erwachsenen mit COPD mithilfe von Musiktherapie zusammen ("`The Effects of Music Therapy in the Treatment of Chronic Obstructive Pulmonary Disease"'). Dies zeigt meines Erachtens die Aktualität dieses Themas und lässt hoffen, dass Musiktherapie evtl. in naher Zukunft in die Standardtherapie bei COPD integriert wird. Abgesehen davon bleibt zu hoffen, dass sich die Situation hinsichtlich der psychosozialen Begleitung und Beratung von COPD-Betroffenen in den nächsten Jahren verbessert und sie frühzeitig eine ihren Bedürfnissen entsprechende Unterstützung erhalten.

Die Auseinandersetzung mit dem Thema "`Musiktherapeutische Stimmarbeit"' als solcher hat mir einmal mehr das Potenzial aufgezeigt, welches in der bewussten und achtsamen Auseinandersetzung mit unserem Körper, Atmung und Stimme liegt.
Diese Aspekte sollten m.E. auch in der musiktherapeutischen Arbeit mitgedacht und eingebettet werden. Um jedoch Patienten diesen Erfahrungsbereich eröffnen zu können, bedarf es zuvor der intensiven Beschäftigung mit der eigenen Stimme. Denn erst, wenn ich selbst gelernt habe, meine Stimme in mir frei und angenehm schwingen zu lassen, kann mich der Stimme anderer zuwenden \autocite[vgl.][15]{mcmurtry2012}. 
Dies hängt unter anderem mit dem in Kapitel \ref{scham_und_stimme} beschriebenen Resonanzphänomen, der klanglich-organismischen-Resonanz, zusammen: nur über die ausgiebige Auseinandersetzung mit und das Erleben der eigenen Stimme können Anspannungen oder Besonderheiten in der Stimme des Klienten wahrgenommen und nachvollzogen werden, wenn sie von der eigenen "`stimmlichen Verfassung"' abweichen. 
So können auf diesem Wege "`in seinem Stimmklang eine Enge oder Verspannung, Energielosigkeit, Druck oder Unbewusstheit in bestimmten Körperzonen" \autocite[16]{mcmurtry2012} entdeckt werden. 
Ein entspannter, selbstverständlicher und freier Umgang mit der eigenen Stimme erleichtert es unseren Klienten zudem, auch ihre Stimme als ganz individuelles Ausdrucksmittel zu nutzen.

Aus den hier zusammen getragenen Gründen erscheint mir die Einbettung dieses Themas in musiktherapeutische Theorie und Forschung als auch in die musiktherapeutische Ausbildung als sehr wichtig. Die Entwicklungen der letzten Jahre geben Grund zur Hoffnung, dass sich das Singen sowohl als musiktherapeutische Methode als auch gesamt-gesellschaftlich wieder mehr etabliert.


\newpage\thispagestyle{empty}
% ---------------------------------------------------------------------------
%: ----------------------- end of thesis sub-document ------------------------
% ---------------------------------------------------------------------------