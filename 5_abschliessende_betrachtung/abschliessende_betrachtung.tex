\chapter{Schlussbetrachtung und Ausblick} % top level followed by section, subsection
Für die erläuterten konzeptionellen Überlegungen im vorherigen Kapitel war es wichtig, sich im Vorherein ausgiebig mit den Themen COPD und musiktherapeutische Stimmarbeit auseinanderzusetzen, um ein grundlegendes Verständnis und Wissen für beide Bereiche entwickeln zu können. So wurde beispielsweise erst im Verlauf der Bearbeitung dieser Masterthesis immer deutlicher, dass der Suchtaspekt in den konzeptionellen Überlegungen aufgegriffen und mitgedacht werden muss. In wie weit nun aber die bereits bestehende Suchtstruktur ursächlich mit der Ausbildung einer Depression oder Angststörung zusammenhängt oder aber tatsächlich die COPD-Erkrankung erst zu dieser Entwicklung führt, kann hier nicht beantwortet werden, stellt jedoch meines Erachtens ein sehr interessantes Forschungsfeld dar. Im Zusammenhang mit einer eventuell bestehenden Ich-Schwäche, wie sie im Suchtkontext vermutet wird, stellt m.E. das Medium "`Stimme"' eine gute Möglichkeit dar, um die Ich-Funktionen zu stärken. Denn kein anderes Instrument todo Ich-Ausdruck

Die Idee des Einsatzes musiktherapeutischer Stimmarbeit als Begleitbehandlung bei einer COPD-Erkrankung beruht auf der Überlegung, die ganzheitliche Wirkung der Stimmarbeit auf körperlicher und psychischer Ebene sich auch in diesem Bereich zu Nutzen zu machen. Gerade in Bezug auf ein gestörtes "`Körperselbst"', wie es weiter oben bereits im COPD-Kontext erläutert wurde, könnte die Form der musiktherapeutischen Arbeit einen positiven Einfluss auf und Veränderungen für dieses bedeuten: Im stimmlichen Ausdruck liegt die Chance, sich selbst als Ganzes zu erleben und wahrzunehmen sowie einen positiven Körperbezug zu stärken. Auch die soziale Funktion des Singens scheint mir im therapeutischen Kontext mit COPD-Betroffenen ein wichtiger Aspekt hinsichtlich der belegten Tendenz zu sozialer Isolation, denn mithilfe der "Brückenfunktion der menschlichen Stimme zwischen "`Innen"' (...) und "`Außen"' können Teilnehmer sich im Kontakt mit anderen erleben und so ihr "`soziales Selbst"' stärken. Dieser Aspekt hat nicht nur Einfluss auf die psychische Verfassung, die Lebensqualität und das Wohlbefinden Betroffener, sondern auch auf die Mortalität. Wie eine Studie englischer Forscher um Andrew Steptoe im vergangenen Jahr zeigte, erhöht "`soziale Isolation"' die Mortalität \autocite[vgl.][]{pmid23530191}. So stellt die Reduzierung sozialer Isolation auch einen wichtigen Aspekt zur Senkung der Sterblichkeit dar.


%Die Wirksamkeitsüberprüfung könnte sich jedoch als schwierig herausstellen, da die Patienten natürlich nicht von anderen wichtigen Therapien, wie z.B. Atemtherapie, ausgeschlossen werden können und somit . Es könnte jedoch über einen längeren Zeitraum 
%Schwierigkeit: unterschiedliche Schweregrade, Hintergründe, Erfahrungen mit dem Singen... es müssten also recht viele Daten erhoben werden. In einem ersten Schritt könnten jedoch sowohl an Teilnehmer dieses Gruppenangebots als auch an nicht teilnehmende, jedoch ebenfalls die pneumologische Rehabilitation in Anspruch nehmende Patienten Fragebögen ausgegeben werden, welche sich dem Thema "`Lebensqualität"' und "`Selbstwirksamkeit"' zuwenden, um einen Teil der Zielsetzungen bereits zu überprüfen. Hierfür gibt es bereits standardisierte Fragebögen, wie sie beispielsweise von Gunter Kreutz und Steven Clift in den zuvor genannten Studien eingesetzt wurden.

Wie bereits zuvor anklang, bestünde nun der nächste Schritt in der praktischen Umsetzung dieses Konzepts. Hierfür würde sich m.E. im Hamburger Raum besonders die "`Atem-Reha"' am Berliner Tor anbieten, da hier das Setting der oben mehrfach beschriebenen pneumologischen Rehabilitation gegeben wäre. So können bereits die praktischen Erfahrungen zu einer Weiterentwicklung des Konzepts führen. Darüber hinaus bedarf es hierfür der weiteren Diskussion und Erfahrungsaustausch mit Kollegen, die ebenfalls in diesem Umfeld tätig sind, sowie schließlich der wissenschaftlichen Überprüfung des Konzepts in der Praxis hinsichtlich der beschriebenen Zielsetzungen. Wie auch Clift et al. (siehe Kapitel \ref{copd_in_der_singforschung}) denke ich, dass in einem ersten Schritt überhaupt getestet werden sollte, ob eine größere Untersuchung hinsichtlich der erzielten Effekte überhaupt sinnvoll erscheint. Für die Überprüfung der Teilziele "Steigerung der Lebensqualität und Selbstwirksamkeit" könnten bereits entwickelte, standardisierte Fragebögen eingesetzt werden, welche bereits von Clift et al. und Kreutz eingesetzt wurden.

Kurz vor Abgabe dieser Arbeit wurde ich in meinem Ansinnen, ein musiktherapeutisches Konzept für die Arbeit mit COPD-Patienten, sehr gestärkt. Das Beth-Israel-Hospital in New York, welches derzeit mit zu den führenden Forschungsinstituten für Musikmedizin und Musiktherapie zählt, stellt seit Ende Mai 2014 eine Teilnehmerkohorte für die Durchführung einer Studie zur Abklärung der Effekte auf die physische Funktionalität und Lebensqualität von Erwachsenen mit COPD mithilfe von Musiktherapie zusammen ("`The Effects of Music Therapy in the Treatment of Chronic Obstructive Pulmonary Disease"'). Dies zeigt meines Erachtens die Aktualität dieses Themas und lässt hoffen, dass Musiktherapie evtl. in naher Zukunft in die Standardtherapie bei COPD integriert wird. Abgesehen davon bleibt zu hoffen, dass sich die Situation hinsichtlich der psychosozialen Begleitung und Beratung von COPD-Betroffenen in den nächsten Jahren verbessert und sie frühzeitig eine ihren Bedürfnissen entsprechende Unterstützung erhalten.

\setlength{\epigraphwidth}{7.5cm}
\epigraph{Study the past, if you would divine the future.}{Confucius}

\ifpdf
    \graphicspath{{X/figures/PNG/}{X/figures/PDF/}{X/figures/}}
\else
    \graphicspath{{X/figures/EPS/}{X/figures/}}
\fi

\lettrine{T}{his} dissertation 

\newpage\thispagestyle{empty}
% ---------------------------------------------------------------------------
%: ----------------------- end of thesis sub-document ------------------------
% ---------------------------------------------------------------------------