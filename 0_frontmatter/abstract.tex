\phantomsection
\chapter*{Abstract}
\markboth{\MakeUppercase{Abstract}}{}
\addcontentsline{toc}{chapter}{Abstract}

This paper introduces a custom-built unmanned aerial vehicle, capable of autonomous exploration in urban environments. It consists of a multicopter, an inertial navigation system and two 2D laser range finders. In addition to a description of the hardware architecture and individual components being used, the authors also discuss challenges and problems that arose during its construction as well as optimizations and workarounds applied in the course of its development.

Also presented is the software architecture, with a focus on a novel algorithm capable of generating multiple next best views (NBVs), sorted by achievable information gain. Although being designed for application on airborne platforms in urban environments, it works directly on raw point clouds and thus can be used with any sensor generating spatial occupancy information (e.g. LIDAR, RGBD- or time-of-flight-cameras). To satisfy constraints introduced by real-time operation on UAVs, the algorithm is implemented on a highly parallel SIMD architecture and benchmarked using GPUs from multiple hardware generations, using data from real flights. It is also compared against the previous, CPU-based proof of concept.

As the underlying hardware imposes limitations with regards to memory access and concurrency, necessary data structures and further performance considerations are explained in detail.

Open-source code for this paper is available at http://www.github.com/benadler/octocopter/.