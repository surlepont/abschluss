% this file is called up by thesis.tex
% content in this file will be fed into the main document

%: ----------------------- name of chapter  -------------------------
\chapter{Konzept}
\label{chapter:konzept}
\setlength{\epigraphwidth}{7.0cm}
\epigraph{To study and not think is a waste. To think and not study is dangerous.}{Confucius}

\section{Gedanken zu COPD und Singen}
Wenngleich der Einsatz der (Sing-)Stimme insbesondere im Hinblick auf ein bewussteres und verlängertes (Aus-)Atmen m.E. in diesem Bereich sehr sinnvoll erscheint, so ist gleichzeitig auch Vorsicht geboten. In diesem Bereich kann es zu entzündlichen Vorgängen rund um den Stimmapparat aufgrund der medikamentösen Behandlung und einer geschwächten Immunabwehr kommen. In diesen Fällen ist es notwendig, durch einen Phoniater abklären zu lassen, ob die Stimme der Schonung bedarf oder aber der gezielte und bedachte Einsatz der Stimme zu einer Besserung der Stimmfähigkeit beiträgt \autocite[vgl.][103ff.]{alavi2009}.
%: ----------------------- paths to graphics ------------------------
% change according to folder and file names
\ifpdf
    \graphicspath{{5_konzept/figures/PNG/}{5_konzept/figures/PDF/}{5_konzept/figures/}}
\else
    \graphicspath{{5_konzept/figures/EPS/}{5_konzept/figures/}}
\fi

\lettrine{T}{he} algorithm for

\newpage\thispagestyle{empty}
% ---------------------------------------------------------------------------
%: ----------------------- end of thesis sub-document ------------------------
% ---------------------------------------------------------------------------