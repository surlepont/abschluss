\ifpdf
    \graphicspath{{4_konzept/figures/PNG/}{4_konzept/figures/PDF/}{4_konzept/figures/}}
\else
    \graphicspath{{4_konzept/figures/EPS/}{4_konzept/figures/}}
\fi

%: ----------------------- introduction file header -----------------------
\chapter{Konzeptionelle \"Uberlegungen}
\label{chapter:konzeptionelle_ueberlegungen}
Die vorangegangenen Auseinandersetzungen mit den Themen COPD und Musiktherapeutische Stimmarbeit dienten dem Aufbau einer Basis, um nun auf dieser ein eigenes Konzept entwickeln zu können. 

Dieses Konzept für die musiktherapeutische Stimmarbeit mit COPD-Patienten ist als übungs- und erlebniszentriertes sowie ressourcenorientiertes Gruppenverfahren angelegt, welches auf der Grundlage psychodynamisch orientierter Musiktherapie Aspekte aus der Körper- und Atemarbeit, Achtsamkeitslenkung und Stimmbildung mit einbezieht. Die folgenden Abschnitte sollen dies nun weiter ausführen und erklären. Am Ende des Kapitels wird ein beispielhafter Sitzungsaufbau beschrieben. Um für die Praxis gut vorbereitet zu sein, wurden im Verlauf des Masterarbeits-Prozesses verschiedene, als geeignet erscheinende Übungen gesammelt, welche in Form einer Literaturliste dem Anhang beigefügt sind.

Im Verlauf der Auseinandersetzung mit diesem Masterarbeitsthema wurden in Fort- und Weiterbildungen (Singende Krankenhäuser, ein atemtherapeutisches Seminar bei Matthias Grot, Wochenendseminare bei Astrid Schmid zur Methode "`Atem-Tonus-Ton"') sowie Selbsterfahrungsseminaren etliche Übungen und Ideen für die praktische Arbeit gesammelt. Da die schriftliche Ausführung dieser Sammlung jedoch den Rahmen dieser Arbeit sprengen würde, wurde dem Anhang eine Liste geeigneter Literatur zu diesem Thema beigefügt.

\section{Psychodynamische \"Uberlegungen}
\label{psychodynamische_ueberlegungen}
In Kapitel \ref{chapter:copd} wurde bereits dargelegt, dass die Diagnose einer COPD auch auf psychischer Ebene für Patienten mit Veränderungen hinsichtlich der möglichen Ausbildung psychischer Komorbiditäten verbunden sein kann. In welchem Ausmaß dies geschieht und mit welchen Folgen es verbunden sein kann, ist individuell abhängig von den jeweiligen biographischen Hintergründen und den damit verbundenen Entwicklungsmöglichkeiten hinsichtlich einer stabilen, gefestigten psychischen Verfassung sowie mit einem unterstützenden sozialen Umfeld.

Wie bereits unter 2. näher erläutert, wird eine COPD nur unter bestimmten Umständen ausgebildet: genetische Veranlagung, Umweltverschmutzung und Einatmen von Schadstoffen bspw. im beruflichen Umfeld sowie zum größten Teil durch einen länger andauernden Nikotin-Abusus (betrifft ca. 80\% der Patienten). 

Was bedeutet aber letzteres nun für die therapeutische Behandlung dieser Patienten? Meines Erachtens scheint es wichtig, diese Suchtthematik mitzudenken und aufzugreifen, da damit suchtspezifische psychische Aspekte verbunden sind. Diese können Besonderheiten insbesondere in der Gegenübertragung mit sich bringen, wodurch die Auseinandersetzung mit diesem Thema seitens des Therapeuten für die Behandlung abhängiger Patienten unumgänglich erscheint. Es ist mir natürlich bewusst, dass mit dieser Erklärung nicht alle potentiellen Patienten eingeschlossen sind. Aber aufgrund der hohen Relevanz der Nikotinabhängigkeit in Verbindung mit der Ausbildung einer COPD bedarf es m.E. des Einbezugs dieser Problematik in therapeutische Konzepte. Es gilt natürlich in der Praxis diese Annahmen stets zu überprüfen und bei Bedarf anzupassen bzw. individuell zu verändern.

Bevor jedoch auf spezielle psychodynamische Aspekte eingegangen wird, bedarf es einer kurzen definitorischen Erläuterung, was unter dem Phänomen Sucht im deskriptiven Sinne verstanden werden kann. 

\begin{quote}
\onehalfspacing
"`Das süchtige Verhalten besteht in dem anhaltenden, starken, unwiderstehlichen Drang, bestimmt durch Drogen bzw. andere Substanzen oder auch durch Tätigkeiten [...] hervorgerufene innere Zustände und Befindlichkeiten von Entspannung oder Anregung immer wieder aufzusuchen oder herbeizuführen."' \autocite[173]{mentzos2011} 
\end{quote}

Betroffene tendieren dazu, im Verlauf die Suchtmittel-Dosis immer mehr zu erhöhen. So sind die Übergänge zwischen normalem Gebrauch und Missbrauch des Suchtmittels bis zur psychischen und körperlichen Abhängigkeit fließend \autocite[vgl.][173]{mentzos2011}. Im ICD-10 findet sich das Störungsbild der Tabakabhängigkeit unter F17 im Bereich der psychischen Störungen; auch das DSM-IV beschreibt das Störungsbild in seinem Diagnostischen und Statistischen Manual Psychischer Störungen. Um eine Abhängigkeit diagnostizieren zu können, müssen mindestens drei von insgesamt sechs im ICD-10 beschriebenen Kriterien seit 12 Monaten vorherrschen. Diese umfassen:

\begin{itemize}
\item "`starkes Verlangen oder eine Art Zwang, Substanzen oder Alkohol zu konsumieren
\item verminderte Kontrollfähigkeit
\item körperliches Entzugssyndrom
\item Toleranzentwicklung (Dosissteigerung)
\item Vernachlässigung anderer Interessen
\item anhaltender Substanz- oder Alkoholkonsum trotz Nachweis schädlicher Folgen (körperlich, psychisch, sozial)"' \autocite[315]{moeller2009}
\end{itemize}

Für die Entstehung und Aufrechterhaltung der Sucht dürfen natürlich die neurobiologischen Prozesse nicht außer Acht gelassen werden. Es wird davon ausgegangen, dass "`es sich bei der stoffgebundenen Sucht um die neurochemische Anpassung des Gehirns an eine anhaltende Substanzzufuhr"' \autocite[14]{tretter2008} handelt. Den Antrieb für das süchtige Verhalten bildet ein neurochemisch begründbarer Belohnungseffekt des Suchtstoffes. Dieser Aspekt ist m.E. selbst aus psychodynamischer Sicht wichtig, da es hier vermutlich zu einem Wechselspiel dieser unterschiedlichen Faktoren kommt und sich aus einer vormals psychischen Abhängigkeit oftmals eine körperliche entwickeln kann.
Darüber hinaus bestehen in Bezug auf die Hintergründe der Suchtentwicklung noch weitere theoretische Modelle (Verhaltenstherapie, Stress-Konzept u.a.) \autocite[vgl.][38ff.]{tretter2008}, auf welche jedoch an dieser Stelle nicht weiter eingegangen werden kann.
 
Folgt man psychoanalytischen Theorien zur Suchtentwicklung, kann davon ausgegangen werden, dass bei Menschen mit einer ausgeprägten Suchtproblematik in der frühen Kindheit ein wichtiger Erfahrungsbereich nicht ausreichend genährt wurde (\cite{mentzos2011}, \cite{weidlinger2012} und \cite{ermann1999}): ein Gefühl für die eigenen Bedürfnisse ist nicht ausreichend entwickelt. Hierdurch ist es auch erschwert, die Wichtigkeit des Einsatzes für sich selbst zu erkennen und eigenverantwortlich zu handeln. Denn dafür bedarf es der frühen Erfahrung, dass ein Kind im Außen diese auf sich bezogene und abgestimmte Zuwendung und Wertschätzung erfahren hat. Austauschprozesse zwischen dem Kind (dem Selbst) und dem Anderen (dem Objekt) im zuvor genannten Sinne, wodurch das heranwachsende Kind "`gute"' Objekte internalisieren kann, sind essentiell für die Entwicklung einer stabilen Persönlichkeit. Dieses Postulat entspringt primär der Objektbeziehungstheorie im Sinne Melanie Kleins. In Bezug auf die Suchterkrankung wird hier davon ausgegangen, dass es sich immer um eine Beziehungskrankheit handelt, welche vermutlich ihren Ursprung in der frühen Kindheit im Übergang von der Symbiose zur Individuation hat \autocite[vgl.][9]{weidlinger2012}. Kann das Kind aufgrund wiederholter Ablehnung, mangelhafter Feinfühligkeit oder gar wegen an ihm ausgeübter körperlicher oder seelischer Gewalt kaum positive Selbstobjekte ausbilden, treten an deren Stelle negative Selbstobjekte. So wird das Kind vermutlich im weiteren Verlauf innerlich mit ambivalenten, polarisierenden Gefühlen sowie mit der Schonung oder aggressiver Wut auf die Objekte beschäftigt sein\autocite[vgl.][10]{weidlinger2012}. 

Diese Ambivalenz zeigt sich ebenfalls in der Sucht, da das Suchtmittel als Beziehungsobjekt sehr ambivalent besetzt ist. "`Auf der einen Seite wirkt es tröstend, beruhigend, entängstigend oder berauschend und anregend; auf der anderen Seite aber bringt es dem Süchtigen Leid, Schuldgefühle, körperliche und seelische Zerstörung bis hin zum Tod"' \autocite[175]{mentzos2011}. Jedoch zeigt sich hierin m.E. primär die "`verinnerlichte Tendenz zur Selbstabwertung wie Selbstzerstörung"' \autocite[10]{weidlinger2012}. Dieser Aspekt, des mangelnden Selbstwertgefühls aufgrund einer strukturellen Störung, findet sich in den psychodynamischen Theorien der Sucht immer wieder. 

So wie beispielsweise in der Sucht-Theorie der Selbst-Psychologie nach Kohut, wird dem Suchtmittelkonsum die Funktion einer Ersatzbefriedigung hinsichtlich mangelhaft genährter narzisstischer Bedürfnisse zugeordnet. Auch Mentzos (2011) schreibt, dass "`das Suchtmittel(...) zum Mittel der notdürftigen Kompensation einer gestörten Selbstwertgefühlregulation"' werde, somit als Ersatz für ein stützendes narzisstisches Selbstobjekt dient. So kann das Suchtverhalten auch als Versuch des Betroffenen verstanden werden, die eigene, innere Sicherheit mithilfe des Suchtmittels wieder herzustellen, welche zuvor durch Enttäuschungen, Kränkungen, Vernachlässigung oder tiefgreifende Konflikte verloren gegangen war. Tress zufolge handelt es sich bei der Sucht um einen Versuch der "`Selbstheilung"'(\cite{tress1985} zitiert in \cite[222]{ermann1999}).

Folgt man diesen Überlegungen, so kann davon ausgegangen werden, dass bei Menschen mit einer entwickelten Suchtproblematik eine erhöhte Vulnerabilität in Bezug auf ihr Selbstempfinden besteht. Dieser Aspekt könnte meines Erachtens in Bezug auf die Krankheitsbewältigung einer ausgebildeten COPD wichtig sein, wie im Folgenden näher erläutert wird.

Bei Daniel Stern finden sich sehr essentielle Vorstellungen für die Entwicklung eines Selbstempfindens wieder. Diese sind jedoch nicht nur in Bezug auf die kindliche Entwicklung von großer Wichtigkeit, sondern bleiben es für das gesamte Leben. Stern unterscheidet hierbei zwischen sechs verschiedenen Bereichen, welche letztlich das Selbst des Menschen bilden: das auftauchende Selbst, das Kern-Selbst, das intersubjektive Selbst, das verbale Selbst sowie mittlerweile das narrative Selbst. Dabei gliedert er seit ein paar Jahren die Empfindung eines Kern-Selbst in zwei Bereiche: zum Einen in die Empfindung eines Kern-Selbst mit seinen drei Invarianzen von Urheberschaft, Kohärenz und Kontinuität und zum Anderen in die Empfindung eines Kern-Selbst in Gemeinschaft mit dem Anderen.  Wie in der Grafik ersichtlich wird, haben die ersten 3 Bereiche nach Sterns neueren Erkenntnissen ihren Ursprung bereits in der pränatalen Phase. \footnote{Da an dieser Stelle nicht auf alle Bereiche des Selbstempfindens ausführlich eingegangen werden kann, empfiehlt sich ein Blick in Daniel Sterns Werk (2007) "`Die Lebenserfahrung des Säuglings"', in welchem der Entwicklungspsychologe, Säuglingsforscher und Psychoanalytiker in sehr anschaulicher Weise die Entwicklung des Säuglings anhand der Entwicklung eines umfassenden Selbstempfindens erläutert.}


\begin{figure}
 \centering
  \includegraphics[width=0.7\textwidth]{selbstempfindungen}
  \caption{Die sechs Selbstempfindungen nach Daniel Stern}
  \label{fig:selbstempfindungen}
\end{figure}


In Bezug auf eine entwickelte COPD scheint es naheliegend, sich mit der Entwicklung eines Kern-Selbst und hier insbesondere mit der Invarianz der Urheberschaft zu beschäftigen, wobei nicht außer Acht gelassen werden darf, dass alle Bereiche ineinander greifen und sie letztlich nur theoretisch auseinanderdividiert werden können.

Warum scheint jedoch eine besondere Auseinandersetzung mit der o.g. Invarianz in Bezug auf eine COPD sinnvoll?

Das Empfinden eines Kern-Selbst entsteht aus dem Zusammenspiel der zuvor genannten Invarianzen der Selbsterfahrung, d.h. unveränderlicher Erfahrungsmuster, die zu einer Organisation und Struktur innerer Prozesse führen. Wie aus der Grafik hervorgeht, handelt es sich um Entwicklungsbereiche, die sich hier natürlich auf den Lebensanfang beziehen, jedoch die Grundlage jeglicher Selbstempfindung bilden. So können sich die einzelnen Bereiche immer weiter ausdifferenzieren, die Basis jedoch bleibt bestehen. 
Stern beschreibt das "`Kern-Selbst-Empfinden […] (als) ein erfahrungsgeleitetes Empfinden von Vorgängen, das wir normalerweise als völlig selbstverständlich voraussetzen und uns nicht bewusst machen. […] Das Selbstempfinden ist kein kognitives Konstrukt; es ist die Integration des Erlebens."' \autocite[106f.]{stern2007} Er bezeichnet es sogar als "`die Grundlage für alle differenzierten Selbstempfindungen, die sich später entwickeln werden"' \autocite[106f.]{stern2007}. 

Was passiert jedoch, wenn ein Mensch spürt, dass er zuvor selbstverständliche körperliche Prozesse, wie im Fall der COPD die Atmung, nicht mehr so steuern kann, wie er möchte? Dies spricht m.E. in erster Linie die Invarianz der Urheberschaft an. Sie umfasst sowohl das Empfinden, Urheber eigener Handlungen zu sein, als auch Nicht-Urheber der Handlungen anderer. Dies ist stets verbunden mit einem willentlichen Vorgehen, der propriozeptiven Wahrnehmung als auch dem Wissen, dass dieses Vorgehen bestimmte Konsequenzen nach sich zieht \autocite[vgl.][106, 114f.]{stern2007}. Im Falle der pneumologischen Veränderungen bei einer COPD, welche stets einhergehen mit ansteigender Atemnot, scheint dieses Gefüge nun ins Wanken zu kommen: die Atmung als "`selbstverständlicher"' und meist nicht bewusst gesteuerter Vorgang verändert sich und dem COPD-Patienten scheint die Kontrolle über diesen Vorgang in manchen Situationen zu entgleiten. Dabei gerät jedoch primär der erste Teil dieser Invarianz, das willentliche Vorgehen, in eine unsichere Position, während die propriozeptive und im Falle der COPD auch die viszerozeptive Wahrnehmung intakt ist und eine Konsequenz für diese körperlichen Vorgänge erahnt werden kann. Dies kann verständlicherweise zu Unsicherheit führen. Bei Menschen, die über ein stark ausgebildetes Ich, ein haltendes und unterstützendes soziales Umfeld sowie über ausreichende Copingstrategien verfügen, wird der Bedarf an professioneller therapeutischer Begleitung vermutlich nicht so hoch sein, wie bei oben beschriebener Klientel, welches aufgrund einer frühen mangelnden bzw. adäquaten Zuwendung nicht über die genannten Ressourcen verfügen. 

Wiederholt sich dieser Vorgang stetig, kann es zu einer Schwächung regulierender Selbstobjekte führen, welche bereits im Säuglingsalter durch die Interaktion mit einem selbstregulierenden Anderen beginnen, sich herauszubilden \autocite[vgl.][338f.]{stern2007}. Je nach individueller Ausprägung der regulierenden Selbstobjekte kann dieser Vorgang früher oder später in eine Regression des Patienten münden. Nun wird die Regulierung auftauchender affektiver Zustände im Außen wieder wichtiger. 
An dieser Stelle kann an eine frühe Erfahrungswelt im Rahmen eines therapeutischen Settings angeknüpft werden, wenn ein geschützter, haltender, stützender und/oder nährender Rahmen geschaffen werden kann \autocite[vgl.][58ff.]{timmermann2008}. Hier eignet sich die Arbeit mit der Stimme besonders. Da die Stimme der primären Bezugspersonen am Anfang des Lebens i.d.R. verbunden wird mit der Erfahrung eines geschützten, nährenden Raums, "`werden wir [lebenslang] in den Tiefen unseres Unbewussten mit Stimmausdruck eine heile Welt assoziieren"' \autocite[282]{deckervoigt2000}. Das "`Heil"' bezieht Decker-Voigt in diesem Zusammenhang darauf, dass uns der Klang der Stimme an eine (intrauterine und frühkindliche) Zeit erinnere, in der wir Kränkungen, Beängstigungen und Verletzungen seelisch noch ertragen konnten \autocite[vgl.][282]{deckervoigt2000}. 

Darüber hinaus scheint in Bezug auf eine angstfreie Ausbildung eines Kern-Selbst auch ein sicherer, haltender Rahmen notwendig. Greift man zurück auf die Erfahrungen des Säuglings, so wissen wir, dass die Entwicklung stets gekoppelt ist an die Verfügbarkeit der primären Bezugspersonen und ihrem Umgang mit dem Säugling. Für eine gelingende Entwicklung ist es wichtig, dass die primären Bezugspersonen (i.d.R. Mutter und Vater) feinfühlig auf das Kind eingehen und als sichere emotionale Basis für das Kind verfügbar sind (Begriffe aus der Bindungstheorie nach J. Bowlby und M. Ainsworth, siehe \cite{brisch2013}) sowie durch Synchronisationsprozesse mit dem Säugling zur Ausbildung einer stabilen psychischen Struktur beitragen. Da zu Beginn des Lebens noch nicht die Möglichkeit zur Selbstregulation gegeben ist, sind auch für diesen Funktionsbereich die primären Bezugspersonen von großer Wichtigkeit. Wird der Säugling mit dieser Überstimulierung durch unbekannte Reize allein gelassen und kann seine eigene Gefühlswelt nicht selbst regulieren, wird er sich vermutlich ängstlich zurückziehen. Hat er jedoch im Außen ein (markiert) spiegelndes \autocite[vgl.][153]{fonagy2004}, feinfühliges Gegenüber, so kann er nach und nach diese nun sich ausbildenden Repräsentanzen in seine psychische Struktur integrieren. 
Übertragen auf die Situation eines erwachsenen Menschen mit COPD kann dies bedeuten, dass er für die Bewältigung seiner gesundheitlichen Krise und zur Prävention vor komorbiden psychischen Störungen von einem regulierenden Anderen profitieren würde. Häufig jedoch sind die näheren Angehörigen aufgrund eigener Involviertheit nicht in der Lage, diesen stützenden Part zu übernehmen oder die Beziehung ist aufgrund der Krankheitssituation bereits zu sehr belastet. 

Daher kann es hilfreich sein, außerhalb der gewohnten sozialen Bezüge einen Raum für sich in Anspruch nehmen zu können, in dem es um die eigene Person geht, so wie sich zu Beginn des Lebens in einem geschützten Rahmen die Handlungen der Bezugspersonen am Säugling orientieren. Im Rahmen einer tiefenpsychologisch fundierten Musiktherapie, wie sie in dieser Arbeit vertreten wird, steht stets der "`Musik erlebende und sich durch Musik ausdrückende Mensch als Klient (im) Zentrum der Aufmerksamkeit"' \autocite[4]{timmermann2004}. Für den Umgang mit der Erkrankung bringt jeder vor dem Hintergrund seiner individuellen Lebensgeschichte Copingstrategien und Ressourcen mit, die ihm helfen, die Situation zu bewerkstelligen. In manchen Fällen ist es jedoch sinnvoll, sich dieser gewahr zu werden, zu verstehen, wie und aus welchen Situationen diese entstanden sind und Handlungsalternativen auszuprobieren. 

\section{Therapeutische Grundhaltung} 
Wie im vorherigen Kapitel ersichtlich, bildet das psychoanalytisch-tiefenpsychologische Verständnis von seelischen Prozessen sowie entwicklungspsychologisches Wissen die Grundlage meiner Arbeit. Meine Grundhaltung ist jedoch mit den Jahren durch die Auseinandersetzung mit anderen Schulen (u.a. während meines Sozialpädagogikstudiums) gewachsen und beinhaltet daher Schulen-übergreifende Aspekte. 

Für die therapeutische Arbeit erachte ich den Aufbau einer haltenden, vertrauensvollen Beziehung, welche von Respekt und Achtsamkeit geprägt ist, als wesentlich. 
Im Rahmen dieser ist es die Aufgabe des Therapeuten, empathisch und kongruent auf sein Gegenüber einzugehen, stets dessen Individualität und subjektive Wahrnehmung akzeptierend. 
Es geht für mich um Begleitungs- und Verstehensprozesse, welche die unterschiedlichen Ebenen der therapeutischen Arbeit, Übungs-, Erlebnis- und Konfliktzentrierung, je nach Kontext und Bedarf nutzen, um darüber hinaus mit dem Patienten angemessene Veränderungs- bzw. Weiterentwicklungsprozesse anzustoßen und voranzubringen. 

In der Arbeit mit COPD-Patienten scheint es m.E. sinnvoll, die Arbeit vorerst primär übungs- und erlebniszentriert sowie ressourcenorientiert auszurichten, da aufgrund begrenzter Zeit (siehe Kapitel \ref{section:gedanken_zum_setting}) und vermutlicher Abwehr gegenüber einem psychotherapeutischen Verfahren eine tiefere Bearbeitung bestehender Konflikte nicht sinnvoll erscheint, insbesondere im Hinblick auf die Erreichung der Therapieziele. Nicht sinnvoll daher, weil ein weiteres Auffangen und Bearbeiten eventuell aufgrund des Settings (siehe Kapitel \ref{section:gedanken_zum_setting}) nicht mehr möglich ist. Sollte es jedoch im Verlauf der Behandlung sinnvoll oder gar notwendig erscheinen, auf solche tiefer einzugehen, ist dies natürlich nicht ausgeschlossen. Allerdings ist es hier erforderlich, immer wieder zu hinterfragen, ob dies tatsächlich für den Patienten momentan tragbar oder sogar überfordernd ist. Insbesondere dann, wenn dies in der Gegenübertragung spürbar wird. Dieser Aspekt ist m.E. für die gesamte Behandlung wichtig, egal auf welcher Ebene gearbeitet wird. COPD-Betroffene sind meist primär mit ihrer Erkrankung beschäftigt und eine bewusste Aufarbeitung psychischer Konflikte könnte schnell destabilisierend wirken, da bereits der Prozess der Krankheitsbewältigung teilweise sehr kraftraubend wirkt. Zu dieser Einschätzung komme ich durch persönliche Gespräche und Erfahrungsberichte von Betroffenen, welche ich im Verlauf der Auseinandersetzung mit der Arbeit geführt oder gehört/gelesen habe. 

Eventuell ergibt sich aber im Anschluss an eine rehabilitative Maßnahme eine längerfristige ambulante Therapie, welche wiederum neue Möglichkeiten eröffnet. 

Darüber hinaus scheint mir für die therapeutische Arbeit mit dieser Klientel, wie zuvor erläutert, der Sucht-Aspekt sehr wichtig. Ich vermute, dass sich in der Behandlung dieser Patienten die Abhängigkeitsthematik auch in der Beziehung zum Therapeuten zeigen kann. So könnten einerseits die unbewussten Wünsche nach Ich-verstärkenden Reaktionen bzw. auch das Gegenteil, wie oben beschrieben, im Sinne der Selbstzerstörung an den Therapeuten herangetragen werden. Hier gilt es in der Gegenübertragung sehr aufmerksam zu sein und diese Tendenzen nicht auszuagieren, sondern vielmehr Interventionen anzubieten, welche die Selbstwirksamkeit des Einzelnen stärken und ihn auf sich zurückführen. Gerade bei Patienten mit einer körperlichen Erkrankung scheint mir hier die Aufmerksamkeitslenkung auf den Körper und die Atmung sehr hilfreich. 

\section{Therapieziele}
Hier kann an die vorherigen Ausführungen gut angeschlossen werden. Eines der mir am wichtigsten erscheinenden Ziele ist die Steigerung der Selbstwirksamkeit. Dieser Gesichtspunkt knüpft sowohl an das Sternsche Thema der Urheberschaft als auch an den unter Kapitel \ref{psychische_komorbiditaet} beschriebenen Circulus Virtuosus an. Die Erfahrung, das eigene Befinden selbst beeinflussen und verändern zu können, kann im Rahmen musiktherapeutischer Stimmarbeit, wie sie hier konzipiert ist, einen wesentlichen Beitrag zur Förderung eines selbstwirksamen Erlebens bieten. Atemvertiefung, sensibilisierte Körperwahrnehmung, Singen als (eventuell wiederentdeckte) Ausdrucksform, die Einbindung in eine und der Austausch innerhalb einer Gruppe sowie positive Beziehungs- und Selbsterfahrungen im Rahmen der therapeutischen Beziehung sind m.E. weitere Aspekte, welche zur Stärkung dieses Bereiches beitragen können. %Aber auch der Aspekt der Resonanz scheint mir in diesem Zusammenhang erwähnenswert. Um sich selbst als Urheber eigener Handlungen wahrnehmen zu können, bedarf es … körperlich wahrnehmbar, verbunden…

Mit Fortschreiten der Erkrankung nimmt in der Regel auch die (subjektive) Lebensqualität Betroffener immer mehr ab. Aus diesem Grund scheint es mir als eine wesentliche Aufgabe, die Steigerung der Lebensqualität auch als Ziel in dieses Konzept aufzunehmen. Ein wichtiges Ergebnis neuerer Singforschung, welche sich dem Thema "`Singen und COPD"' widmet (siehe Kapitel  \ref{copd_in_der_singforschung}, zeigt zudem, dass insbesondere die Lebensqualität Betroffener durch eine regelmäßige Teilnahme an einem Singangebot signifikant gesteigert werden kann. Singen als "`natürliches Anti-Depressivum"' könnte zudem einen wichtigen Beitrag zur Unterbrechung des beschriebenen COPD-Teufelskreises bieten, da durch die beiden zuvor genannten Wirkungseffekte eventuell die Wahrscheinlichkeit zur pathologischen Ausbildung einer depressiven oder Angstsymptomatik reduziert werden kann. Diese Annahme gilt es jedoch bei einer späteren Untersuchung in der Praxis zu überprüfen. So enthält dieses Konzept auch die Absicht der Prävention bzgl. der sich oftmals im Verlauf der COPD zeigenden Komorbiditäten wie Depression und Angst.

Für einige mag gerade die akute Gesundheitsverschlechterung zu einem krisenhaften Erleben der eigenen Situation führen. Dies kann durchaus auch mit der Auseinandersetzung der Begrenztheit des eigenen Lebens sowie mit dem bewussten Auftauchen der Todesangst einhergehen. Hier kann es das Ziel musiktherapeutischer Arbeit sein, Patienten einen Ort für die Wahrnehmung, Akzeptanz und den Umgang mit den damit verbundenen Affekten zu eröffnen und sie darin zu begleiten, die Möglichkeiten zum persönlichen Wachstum und zur Weiterentwicklung zu erkennen und zu nutzen, welche jeder Krise innewohnen. Dabei scheint es förderlich, durch den Austausch mit einem oder mehreren Spiel- und Gesprächspartnern und die bewusste Auseinandersetzung mit der bisherigen Selbsteinschätzung und Lebenssituation neue Sichtweisen auf sich und die eigene Lebensbehandlung zu entwickeln. Hier stellt die Musik und insbesondere der stimmliche Ausdruck eine wichtige Ergänzung zur Sprache dar, denn durch den Wechsel vom "`Darüber-Reden"' zum "`Es-Spielen"' kann ein sehr hilfreicher Ausweg aus den herkömmlichen "`Denk- und Fühlschablonen"' geschaffen werden. Hierin enthalten ist zudem das Thema der Krankheitsbewältigung, welches mir für die Arbeit mit dieser Klientel als sehr wesentlich erscheint. Aufgrund der mangelnden psychologischen Betreuung, häufiger sozialer Isolation mit Fortschreiten der Erkrankung sowie seltenem Kontakt zu anderen COPD-Patienten fehlen Betroffenen oftmals Austauschmöglichkeiten über krankheits-immanente Themen und Fragen, wodurch sie in ihrer eigenen Gedanken- und Phantasiewelt verbleiben. Hierfür scheint mir die Arbeit in der Gruppe sehr sinnvoll, wie im folgenden Kapitel näher erläutert werden soll.

\section{Gedanken zum Setting}
\label{section:gedanken_zum_setting}
Die vorangegangene Darstellung der therapeutischen Ziele deutet bereits darauf hin, dass in diesem Kontext die Arbeit in der Gruppe als sinnvoll erscheint. Insbesondere Patienten, die von sozialer Isolation bedroht sind und Schwierigkeiten (entwickelt) haben, sich in sozialen Gruppen zu bewegen, können in einem geschützten und durch die Therapeutin gehaltenen Rahmen neue Erfahrungen sammeln, diese reflektieren und über diesen Prozess wieder Vertrauen in die eigenen Kontaktmöglichkeiten fassen.

Zudem kann es im Austausch mit der Gruppe zur Hinterfragung der eigenen \mbox{Lebens-be-handlung} kommen und alternative Umgangsformen und Lösungsstrategien entwickelt werden. 

Vielen Patienten fehlt oftmals der Kontakt zu anderen Betroffenen und es fällt ihnen schwer, sich aus eigenem Antrieb an Selbsthilfegruppen oder Beratungsstellen zu wenden. Dies kann auch aus einem geschwächten Gefühl für eigene Bedürfnisse und einer verzerrten Selbstwahrnehmung resultieren. Auch hier kann der Einzelne gerade von der Gruppensituation profitieren, da hier ein Raum für den Abgleich von Selbst- und Fremdwahrnehmung geschaffen werden kann.

Gerade für die Krankheitsbewältigung kann der Austausch innerhalb einer diagnosespezifischen Gruppe daher hilfreich und unterstützend wirken. Um jedoch dem oben entwickelten Gedanken in Bezug auf ein haltendes, stützendes und spiegelndes Gegenüber innerhalb eines geschützten Rahmens, in welchem ein Raum zum Ausprobieren, zur Reflexion, Stärkung und Veränderung geboten wird, gerecht zu werden, bedarf es einer therapeutisch ausgebildeten Leitung dieser Gruppe, welche über Wissen zu gruppendynamischen Prozessen und Gruppenphasen verfügt. 

Zudem scheint mir das Gruppensetting hinsichtlich der Suchtthematik als sinnvoll. Während in der Einzeltherapie die Ablösungsschritte vom Therapeuten zugunsten einer größeren Autonomie beschwerlicher sind, können diese durch die Sicherheit gebende Gruppe leichter ausprobiert und gegangen werden \autocite[vgl.][]{nawe2014}. Gerade in Bezug auf die Stärkung der Selbstwirksamkeit ist dies meines Erachtens ein wichtiger Aspekt.

Um diesen geschützten Rahmen aufrechterhalten zu können und gruppendynamische Prozesse für den Therapeuten überschaubar zu gestalten, ist eine Gruppengröße von 5-8 Patienten optimal \autocite[vgl.][]{weber2013}. Um Überforderungstendenzen sowohl auf körperlicher als auch auf geistig-seelischer Ebene entgegenzuwirken, wird eine Sitzungsdauer von 50-60 Minuten angestrebt. Zudem ist eine halboffene Gruppenform angedacht, um einerseits den Gruppenprozess nicht allzu häufig zu stören und gleichzeitig Stagnation entgegenzuwirken, sowie andererseits möglichst vielen potentiellen Patienten den Zugang zu diesem gruppentherapeutischen Angebot zu ermöglichen. 

Während der Auseinandersetzung mit diesem Masterarbeitsthema entstanden unterschiedliche Ideen, in welchem Rahmen eine Durchführung dieses Konzepts möglich erscheint.
Die erste Idee bestand darin, Betroffene über Selbsthilfegruppen, Arztpraxen und Kliniken zu erreichen und ein ambulantes Angebot zu gestalten. Dies würde bedeuten, dass Patienten eventuell einen längeren Weg auf sich nehmen müssten, dies jedoch zum Teil aufgrund ihres Gesundheitszustandes nicht bewerkstelligen können. Da allerdings m.E. regelmäßige Sitzungen mindestens 1-2mal wöchentlich gerade für den Anfang notwendig erscheinen, um mit einer Gruppe im Sinne eines gruppentherapeutischen Konzepts arbeiten und die Erreichung der Ziele realistisch gestalten zu können, ist dieses Setting gerade für den Anfang nicht sinnvoll. Zudem konnte bei einer Kontaktaufnahme mit einer COPD-Selbsthilfegruppe \footnote{Mit dem Ziel, noch mehr Kenntnis über die Erfahrungen und Bedürfnisse von Betroffenen zu erlangen, fragte ich bei dem Leiter einer Selbsthilfegruppe an, ob ich zu einem Treffen am 10.05.2014 in der Asklepios-Klinik-Barmbek hinzustoßen könne. So habe ich meine Intention hinsichtlich der hier vorliegenden Masterarbeit geschildert, wurde jedoch mit dem Hinweis abgewiesen, dass die Teilnehmer sich nicht zur "`Hilfe für andere"' zusammengeschlossen hätten, sondern "`sich selbst Hilfe"' erwünschen. Auch das Angebot, im Gegenzug zu einem späteren Zeitpunkt z.B. einen Vortrag zum Thema zu halten, stellte hier keine Option dar. Zu stark schien die Identifizierung mit der Rolle des Opfers.} viel Widerstand hinsichtlich eines solchen Angebots wahrgenommen werden, was zum einen aus der Sorge resultierte, dass Singen für viele zu anstrengend sei - ich vermute hier jedoch auch einen Zusammenhang mit dem Thema Scham - und zum anderen eine Distanzierung hinsichtlich psychotherapeutischer Begleitung, da Betroffene sich primär auf körperlicher Ebene Unterstützung wünschen. Eine Kollegin berichtete zudem über ähnliche Erfahrungen mit dieser Patientengruppe. 

Daher denke ich, dass dieses Konzept am besten in ein Setting eingebettet wäre, welches musiktherapeutische Stimmarbeit als einen Teil der Behandlung ansähe und sie in die Behandlungspläne der Patienten mit aufnähme. Hier sollte der Zugang so leicht wie möglich gestaltet sein, so dass Patienten keine langen Wege zu den Sitzungen unternehmen müssen, sondern am besten bereits vor Ort sind. Einen Rahmen, in welchem Patienten über mehrere Wochen in eine fortlaufende tägliche Behandlung eingebunden sind, stellt hierbei die pneumologische Rehabilitation dar. Wie bereits unter Kapitel \ref{nicht-medikamentoese_therapien} erläutert, kann diese ambulant, teilstationär oder stationär durchgeführt werden. Die psychotherapeutische Begleitung hat hier jedoch nach wie vor einen sehr geringen Stellenwert, lediglich die Raucherentwöhnungsprogramme werden über den Zeitraum der Rehabilitation in der Regel von einem Psychologen durchgeführt. Gespräche mit einem Psychologen sind meist fakultative Angebote. Ansonsten gehören edukative Vorträge zu einer regulären pneumologischen Rehabilitation dazu, was sicherlich einen großen Stellenwert hinsichtlich der Krankheitsbewältigung hat, jedoch nicht den Einzelnen mit seinen individuellen Unterstützungsbedürfnissen beachtet. 

Aus Beobachtungen während eigener pneumologischer Rehabilitationsmaßnahmen weiß ich, dass sich trotz vorheriger Skepsis die meisten Patienten zusätzlichen Angeboten wie Ergo- oder Kunsttherapie durch das Ausprobieren öffnen konnten und es ihren Aufenthalt bereicherte. So könnte ich mir auch bei diesem Konzept vorstellen, dass durch eine selbstverständliche Einbettung dieses Angebots in die Behandlung eine größere Akzeptanz und die Bereitwilligkeit zur Teilnahme im Vergleich zur ambulanten Arbeit ohne Anschluss an eine solche Institution steigen könnte. 

In Bezug auf den Gedanken eines halboffenen Gruppenangebots (siehe oben) wäre m.E. eine Aufnahme in die Musiktherapie-Gruppe alle zwei Wochen sinnvoll, um zu vielen Wechseln entgegenzuwirken und so eine, dem Kontext mögliche, Kontinuität zu schaffen. Neuaufnahmen finden in der pneumologischen Rehabilitation i.d.R. zwei- bis dreimal am Anfang der Woche statt, so dass die ersten Therapien Mitte/ Ende der Woche beginnen können. Wenn möglich wären aufgrund der geringen Teilnehmerzahl zwei parallel laufende, jedoch versetzt aufnehmende Gruppen, optimal, um Patienten eine Teilnahme über den gesamten Verlauf ihrer Rehabilitationsmaßnahme zu ermöglichen.

Die Anbindung an ein Krankenhaus wurde ebenfalls angedacht, jedoch scheint es auch hier schwierig, eine Kontinuität aufrecht zu erhalten. Aufgrund meist kürzerer, begrenzter Aufenthalte könnten hier Singangebote, Einzeltherapien oder musiktherapeutische Kurzinterventionen angedacht werden. Im Sinne einer Gruppenbehandlung sowie für die angedachte Prozessbegleitung sehe ich eine regelmäßige Teilnahme über mehrere Sitzungen für dieses Konzept aber als wichtig an. 

Eine feste Einbindung in ein rehabilitatives Behandlungskonzept könnte zudem bedeuten, dass ein fester Raum für diese musiktherapeutische Arbeit zur Verfügung stünde und somit der Einbezug von Instrumenten gerade zu Beginn der Behandlung möglich würde, um die anfänglichen Hemmungen besser auffangen zu können. Im ambulanten Setting scheint die Arbeit mit einem größeren Instrumentarium aus logistischen Gründen eher schwierig, es sei denn, es gäbe Lagermöglichkeiten.

Abschließend ist also festzuhalten, dass gerade bei der ersten praktischen Umsetzung dieses Konzepts, welche den nächsten Schritt bedeuten würde, eine Einbindung in eine Einrichtung der pneumologischen Rehabilitation sinnvoll wäre.

\section{Indikation/Kontraindikation}
Wie bereits unter Kapitel \ref{psychische_komorbiditaet} erläutert, manifestieren sich Angst- und Panikstörungen sowie Depressionen bereits in den frühen Stadien der Erkrankung und nehmen in der Regel mit dem Fortschreiten der Erkrankung zu. Daher gilt es hier, sich nicht nach den Schweregraden, sondern nach dem Bedarf an psychotherapeutischer Begleitbehandlung zu orientieren. 

Generell sollte jedoch aufgrund des erhöhten Risikos zur Ausbildung einer psychischen Komorbidität für jeden Patienten, der daran Interesse hat und davon profitieren könnte, die Möglichkeit zur Teilnahme an diesem therapeutischen Behandlungsangebot bestehen. Denn wichtige, oben genannte therapeutische Ziele liegen sowohl in der Prävention zur Ausbildung einer psychischen Erkrankung als auch in der Krankheitsbewältigung. Eine wichtige Kontraindikation besteht m.E. jedoch in einer starken Abwehr gegenüber der hier beschriebenen Stimmarbeit. Patienten, deren Widerstand gegenüber der Nutzung ihrer eigenen Stimme als Ausdrucksmittel zu groß erscheint, sollte bei Bedarf eine alternative Form psychosozialer Unterstützung angeboten werden.

Wie bei jeder gruppentherapeutischen Behandlung gelten auch hier folgende Ausschlusskriterien: akute Psychosen/hirnorganische Störungen, Schwierigkeiten, einen Leiter zu akzeptieren und/oder interpersonelle Beeinträchtigungen. Darüber hinaus sollte die Motivation potentieller Teilnehmer hinsichtlich der Behandlung überprüft werden. Weitere mögliche Kriterien zur Gestaltung einer Gruppe im psychotherapeutischen Setting können bei Strauß und Mattke gefunden werden \autocite[vgl.][78-88]{mattke2007}.

Eine weitere Kontraindikation könnte fehlende Mobilität bedeuten, da Teilnehmer das Gruppenangebot selbstständig oder mit Hilfe aufsuchen müssen. Aber auch Menschen, die auf einen Rollstuhl oder dauerhafte Sauerstoffgabe angewiesen sind, können an diesem musiktherapeutischen Angebot teilnehmen. An dem anschließend beschriebenen Studienprojekt in Kent (siehe Kapitel \ref{copd_in_der_singforschung}) haben ebenfalls Personen teilgenommen, welche auf zusätzliche Sauerstoffversorgung angewiesen waren; diese Personen konnten sich dennoch am Angebot beteiligen. Bei gleichzeitigem Bedarf an psychotherapeutischer Begleitung sollte jedoch bei Patienten, die nicht an dem Angebot teilnehmen können, ein therapeutisches Angebot im aufsuchenden Einzelsetting in Betracht gezogen werden.

Sollte es im Vorgespräch mit einem Patienten Anzeichen für eine Traumatisierung geben, gilt es hier abzuwägen, ob eine Gruppenbehandlung im hier angedachten Sinne angemessen erscheint. Oftmals ist es gerade für diese Patientengruppe wichtig, in einem geschützten und sehr strukturierten Rahmen äußere und innere Sicherheit aufzubauen, um sich dann eventuell freieren Formen anzunähern. Da bei diesen Menschen das "`Erstarren und Verstummen [...] einen überlebensnotwendigen Schutz darstellen"' \autocite[68]{rittner2012} und Musik einen Trigger für Retraumatisierungen darstellen kann, gilt es hier respektvoll und sehr behutsam vorzugehen.

Wenngleich der Einsatz der (Sing-)Stimme insbesondere im Hinblick auf ein bewussteres und verlängertes (Aus-)Atmen m.E. in diesem Bereich sehr sinnvoll erscheint, so ist gleichzeitig auch Vorsicht geboten. Bei Patienten mit COPD kann es aufgrund der medikamentösen Behandlung und einer geschwächten Immunabwehr zu entzündlichen Vorgängen rund um den Stimmapparat kommen. In diesen Fällen ist es notwendig, durch einen Phoniater abklären zu lassen, ob die Stimme der Schonung bedarf oder aber der gezielte und bedachte Einsatz der Stimme zu einer Besserung der Stimmfähigkeit beitragen kann \autocite[vgl.][103ff.]{alavi2009}.

\section{COPD in der Singforschung}
\label{copd_in_der_singforschung}
In der neueren Singforschung gibt es mittlerweile einige Ergebnisse, die darauf hinweisen, dass der Einsatz des therapeutischen Singens speziell für COPD-Patienten aus unterschiedlichen Gründen von großem Nutzen sein kann. Es wurden hier sowohl positive Effekte in Bezug auf physische Parameter (wie z.B. Verbesserung der Lungenfunktion) als auch auf psychosoziale Faktoren (wie die Steigerung der Lebensqualität) gemessen.
In der englischsprachigen textbasierten Meta-Datenbank Pubmed, welche die weltweit umfassendste Datenbank für medizinische und psychologische Artikel darstellt, konnten zum Thema Singen und COPD/Lungenemphysem insgesamt sieben relevante Studien sowie eine aktuelle Literaturrecherche gefunden werden \autocite{pmid19436683,pmid20175359,pmid20682030,pmid23145504,pmid23497924,pmid23497929,pmid24398814,pmid24793633}. Die Ergebnisse sind sehr weit gestreut und lassen keine klaren Aussagen über die Wirksamkeit des Singens auf die physische, funktionale oder psychische Konstitution von COPD-Patienten zu. Dies könnte u.a. dem Umstand geschuldet sein, dass die Studiendesigns noch optimiert werden müssen (zu kleine Teilnehmer-Kohorte, zu kurze Studiendauer, wenige randomisierte und kontrollierte Studien). Jedoch kann aufgrund der in unterschiedlicher Ausprägung primär positiven Effekte davon ausgegangen werden, dass COPD-Patienten von therapeutischem Singen, welches auf diese Patientengruppe hinsichtlich der Übungs- und Liedauswahl abgestimmt ist, profitieren könnten.

Die umfangreichste und bisher am größten angelegte Studie wurde im Zeitraum zwischen September 2011 und Juni 2012 in der traditionellen Grafschaft Kent im Südosten Englands durchgeführt, da in dieser Region die COPD-Prävalenz besonders hoch sei \autocite[vgl.][4]{clift2013} \footnote{Der Titel der Studie lautete: "`A feasibility study on the health benefits of a participative community singing programme for older people with COPD"'. Eine Videozusammenfassung der Studie kann unter folgendem Link angeschaut werden: https://www.youtube.com/watch?v=c0UK2X3i-FU}.

Im vergangenen Jahr hatte ich die Möglichkeit, den Hauptinitiator dieser Studie, Steven Clift, bei einem Kongress der Singenden Krankenhäuser zu erleben und die neuesten Ergebnisse zu diesem Thema aus erster Hand zu erfahren. 

Es handelt sich hierbei um eine relativ groß angelegte Feasibility (Machbarkeits-) Studie (n=109,Durchschnittsalter: 69), jedoch ist das Studiendesign nicht-kontrolliert sowie -randomisiert. Dies entstand aus dem Ansinnen heraus, eine Kohorte zu bilden, die die Motivation mitbringt, an einem Singprogramm über 10 Monate lang teilzunehmen, so dass möglichst viele am Ende hinsichtlich der Studienziele überprüft werden können. Dieses Studiendesign ist insofern wichtig, als daran eine größere randomisierte und kontrollierte Studie angeschlossen werden kann. Die Rekrutierung fand über allgemeinmedizinische Praxen, Community Health-Zentren, Zeitungsausschreibungen und die Britische Lungenstiftung in Kent statt. Die Patienten wurden in insgesamt sechs Gruppen aufgeteilt, welche sich vor Ort einmal wöchentlich über 10 Monate zum professionell angeleiteten Singen trafen. Diese Einheiten gingen über 90 Minuten, worin jedoch mind. 30 Minuten Atem-, Körper- und Stimmübungen enthalten waren. Die Patienten brachten in der Regel keine regelmäßigen Singerfahrungen mit \autocite[vgl.][5ff.]{clift2013}.

Ziel war es, herauszufinden, von welchen Effekten ältere Menschen mit COPD hinsichtlich der Messwerte Atmung, gesundheitsbezogene Lebensqualität sowie der physischen und psychischen Gesundheit durch ein wöchentliches Singangebot in der Gruppe über einen Zeitraum von zehn Monaten profitieren würden. 

Die Teilnehmer wurden am Anfang, in der Mitte (nach 5 Monaten) und am Ende beurteilt und mussten zu diesen Zeiten jeweils Fragebögen zum Thema Lebensqualität (St. George's Respiratory Questionnare, SGRQ) und zum Thema "`Nutzung des Gesundheitssystems"' ausfüllen. Die Lungenfunktionstestung wurde am Anfang und am Ende durchgeführt. 

Die Studienergebnisse zeigen zum Einen eine signifikante (p=0,006) Verbesserung der Lungenkapazität (Steigerung des FEV1 um 2\% nach 10 Monaten) und zum Anderen eine Steigerung der Lebensqualität um drei Messpunkte. Damit entspricht es dem gleichen Wert, welcher auch in Medikamentenstudien mit COPD-Patienten erreicht werden konnte \autocite[vgl.][]{clift2013a}. So berichteten Teilnehmer, dass sich die regelmäßige Teilnahme an der Singgruppe sehr positiv auf ihr soziales Leben sowie ihre physische und psychische Konstitution ausgewirkt hätte \autocite[vgl.][6ff.]{clift2013}.

%\section{Einbezug von Körper und Atem} 
%\label{section:einbezug von koerper und atem}
%Wie bereits zuvor immer wieder erwähnt, stellt der Einbezug von Körper und Atem in der musiktherapeutischen Arbeit mit COPD-Patienten einen wichtigen Aspekt hinsichtlich der angestrebten, beschriebenen Therapieziele dar. Atem- und Stimmübungen, wie sie hier angedacht sind, 

%Da es sich bei der COPD primär um eine körperliche Erkrankung handelt, soll an dieser Stelle noch eine weitere Theorie, die des "`Embodiments"', hinzugezogen werden, die sich auf die Wechselwirkung von Körper und Psyche bezieht. 

%Vier Vertreter unterschiedlicher Disziplinen (Kognitionswissenschaften, Psychologie, Neurobiologie und Körperarbeit) haben zusammengetragen, was aus ihren unterschiedlichen Blickwinkeln zum Zusammenhang von Körper und Psyche wichtig erscheint. Entstanden ist diese Zusammenarbeit aus der gemeinsamen Erfahrung, dass der Zusammenhang von Psyche und Körper in vielen Bereichen noch mangelhaft Beachtung findet und gerade in therapeutischen und beratenden, aber auch wissenschaftlichen Arbeitsfeldern, in deren Fokus der Mensch steht, wichtiger Bestandteil der Betrachtung des Einzelnen sein sollte.
%Storch, Cantieni, Hüther und Tschacher haben jedoch in ihrer Publikation nicht das Rad neu erfunden, sondern bestehendes Wissen und Ideen zum Thema zusammengetragen und weiterentwickelt. Embodiment-Theorien gehen davon aus, dass eine Wechselwirkung "`zwischen allem, was als Körpergeschehen aufgefasst werden kann (dies beinhaltet einzelne motorische Aktionen und Bewegungsabläufe bis hin zu ganzen Verhaltenssequenzen) und dem psychischen System"' \autocite[39]{hüther2010} besteht. todo

%Körper- und Atemwahrnehmung sind jedoch nicht nur zur Unterstützung der Atemwegserkrankung sinnvoll und wichtig, sondern auch Achtsamkeit in Bezug auf die Sucht im Hier und Jetzt sein -> MBSR.

%Schlaffhorst Andersen
%MT und Atem (Buch)
%Annette Cramer
%Rittner
%Hertha Richter

\section{Überlegungen zur praktischen Umsetzung}
In einem geschützten Raum soll die Möglichkeit geschaffen werden, in Kontakt zu gehen, sich auszutauschen, eigenes Erleben zu teilen und zu reflektieren. In der Gruppe Rückhalt zu finden, sich nicht mehr isoliert und alleine mit der Erkrankung zu fühlen, dies jedoch in einem therapeutischen Rahmen, welcher die Chance für die Hinterfragung der eigenen Lebensbehandlung sowie zur Veränderung bietet, sind die Intentionen dieses Konzepts. Abgestimmt auf das Krankheitsbild soll jedoch nicht nur auf der kognitiven Gesprächsebene angesetzt werden, sondern über Körper- und Atemübungen bis hin zum stimmlichen Ausdruck die eigene Körpersensibilität und der Zugang zur eigenen emotionalen Verfassung und ihrem Ausdruck ausgeweitet werden, um dadurch wieder zu mehr Sicherheit und Selbstvertrauen in die eigenen selbstwirksamen Kräfte zu gelangen. Wenn die institutionellen und finanziellen Rahmenbedingungen es ermöglichen, wäre ein Grundinstrumentarium hilfreich, insbesondere in der Anfangsphase, jedoch nicht notwendig. Der Einstieg über die instrumentale Improvisation kann hinweghelfen über die Hemmung des Stimmeinsatzes (wie zuvor bereits erläutert). Sollten Instrumente zur Verfügung stehen, gilt es diese am Anfang der Behandlung einzuführen, um sie jederzeit einbeziehen zu können. Insbesondere auch bei der Bearbeitung von Themen, welche aufgrund ihrer emotionalen Tiefe für den stimmlichen Ausdruck zu intensiv und bedrohlich erscheinen, kann der Einsatz von Instrumenten hilfreich sein. 
Während jede stimmliche Äußerung, ob Sprache oder Gesang, einen unmittelbaren Ich-Ausdruck darstellt, kann durch das Getrenntsein des Instruments von der eigenen Person (anders als bei der Stimme) auch das eigene Spiel von der eigenen Person abgespalten werden \footnote{Hilfreiche, sich selbst beruhigende Erklärungen wie "`ich habe nicht gelernt, auf dem Instrument zu spielen"', "`das ist ja fürchterlich verstimmt"' usw.} und wirkt dadurch weniger exponierend. So bietet das Instrument Halt und Rückzugsmöglichkeit und Patienten können vorsichtig beginnen, sich zu öffnen. Bei Rittner finden sich hierzu deckende Erfahrungsbeschreibungen \autocite[vgl.][111]{rittner1990}.  

%Nachdem unter \ref{psychodynamische ueberlegungen} einige psychodynamische Phänomene bei einer Suchtproblematik zusammengetragen wurden, sollen diese nun auch bei den Überlegungen zum Aufbau und Inhalt der Sitzungen Berücksichtigung finden. 

In Bezug auf das Thema "`Scham"' sowie auf möglicherweise auftretende Widerstände gegenüber dem Singen sollten die Stunden zu Beginn sehr klar strukturiert werden; sowohl vom Aufbau als auch im musikalischen Tun. So stehen am Anfang Körper- und Atemwahrnehmung im Vordergrund. Darauf aufbauend scheint es mir sinnvoll, eventuell mit Rhythmusinstrumenten, dem Körper, Atem und/oder der Stimme rhythmische Übungen einzuführen, um Sicherheit in diesem Kontext aufzubauen. Der stimmliche Ausdruck kann vorsichtig im Rahmen von Atemübungen angebahnt werden. So bietet sich hier beispielsweise die Vokalraumarbeit nach Ilse Middendorf an, bei welcher Vokale vor dem Erklingen imaginiert werden. Bevor die Vokale erklingen, werden sie geatmet, indem die Formung des Vokals primär im Ansatzrohr simuliert, jedoch nicht an den Stimmlippen in Schwingung versetzt und dadurch hörbar wird. Erst wenn der Vokal dem Übenden klar geformt erscheint, wird er langsam vernehmbar. Interessant ist dabei, dass ohne das Ertönen-Lassen der Atem bereits in verschiedene Bereiche fließt, je nach Wahl eines Vokals oder aber auch Konsonanten. Bei Middendorf gibt es hierzu schematische Aufzeichnungen \autocite[vgl.][60ff.]{middendorf1995}. Diese Übungen sind sehr gut geeignet, um die eigenen Atemräume besser zu erkunden und gleichzeitig in einem leistungsfreien Raum das Ertönen-Lassen der eigenen Stimme auszutesten. 

Für den Aufbau eines Sicherheit und Halt gebenden Rahmens ist zudem die Gestaltung von Ritualen hilfreich \autocite[vgl.][31ff.]{deckervoigt2013}. Das Ritual als sicheres Kontinuum stellt m.E. gerade bei Menschen mit COPD einen wichtigen Aspekt vor dem Hintergrund plötzlich auftretender Atemnot und Krankheitsverschlechterung dar. Dies soll sich im Folgenden sowohl in der Struktur des möglichen Sitzungsaufbaus als auch in einzelnen Gesten in der späteren praktischen Umsetzung wiederspiegeln wie z.B. in der Ausgestaltung der Übergänge zu Beginn und am Ende einer Sitzung.

Wie bereits unter Kapitel \ref{section:gedanken_zum_setting} erwähnt, ist es für die Leitung therapeutischer Gruppen notwendig, gruppendynamische Prozesse wahrnehmen und aufgreifen sowie diese gedanklich immer wieder in den Kontext der unterschiedlichen Gruppenphasen bringen zu können. Dies hat wiederum Einfluss auf die eigenen Interventionen z.B. in Bezug auf die Aspekte Anleiten versus Zurückziehen.

Da bisher noch keine praktischen Erfahrungen mit diesem Konzept gesammelt werden konnten, stellen die hier dargestellten pragmatischen Überlegungen Möglichkeiten dar, welche es in der Praxis auszutesten und bei Bedarf anzupassen und zu ergänzen gilt. 

Um die vorherigen Überlegungen nun etwas zu konkretisieren, soll im Folgenden ein möglicher Sitzungsaufbau dargestellt werden. Diese hier erkennbare Struktur wurde vor dem Hintergrund der Besonderheiten der angesprochenen Klientel entwickelt. So ist eine Gliederung der Sitzungen in folgende Teilbereiche angedacht:

\subsection*{Vorbereitung}
Bevor die Gruppenteilnehmer erscheinen, ist es wichtig, sich (mithilfe der eigenen Dokumentation) zu vergegenwärtigen, was in der letzten Sitzung geschehen ist, um an diese anschließen und aktuelle Prozesse in diesem Zusammenhang eventuell besser verstehen zu können sowie der Gruppe dadurch den Einstieg zu erleichtern. Auch das Einholen von Informationen über die Patienten bzw. Prozesse im institutionellen Rahmen können für die Sitzung hilfreich sein, zeigt man u.a. hiermit ein Interesse an den Belangen der Patienten sowie am Geschehen in der Institution. Handelt es sich um die erste Sitzung, kann man die Vorgespräche, welche m.E. unerlässlich vor einer tatsächlichen Gruppenteilnahme sind, noch einmal Revue passieren lassen und sich so auf die Gruppe einschwingen.
Da nach Eintreffen aller Teilnehmer zuerst eine Gesprächsrunde stattfinden soll, bietet es sich an, zuvor einen Stuhlkreis aufzustellen, in welchem die Teilnehmer ihren Platz zu Beginn auswählen können. 

\subsection*{Ankommen}
In diesem Teil geht es darum, einen sanften Übergang vom "`Vorherigen"' in das "`Hier und Jetzt"' zu schaffen. So gibt es nach dem ersten Begrüßen für jeden die Möglichkeit, sich mitzuteilen. Hier können die eigene Stimmung, aktuelle Themen, Belastungen oder auch freudige Erfahrungen geteilt oder auch spezielle Fragen, Wünsche oder Kritik hinsichtlich der Gruppentherapie geäußert werden. An dieser Stelle möchte ich gern ein methodisches Vorgehen einbringen, welches ich bei Thomas Jüchter im psychosomatischen Kontext (Evangelisches Krankenhaus Ginsterhof in Rosengarten) kennengelernt habe \footnote{Herr Jüchter hat hierfür sein Einverständnis gegeben.}. Wenngleich die Einzelnen in dieser Anfangsrunde häufig vorerst unverbunden miteinander erschienen, so ließ sich doch aus ihren Äußerungen ein Gruppenthema herausarbeiten, welches als Überthema in das musikalische Arbeiten der aktuellen Sitzung eingebunden werden konnte. Dieses Thema wird von dem Therapeuten aus den Äußerungen der Einzelnen herausgearbeitet und kann durch die Gruppenteilnehmer noch ergänzt oder verändert werden. So können beispielsweise bei einem Thema wie "`Sich-ausgeliefert-fühlen"' Körper- und Atemübungen aber auch themenbezogene Lieder auf die eigenen selbstwirksamen Kräfte hinführen. Bei einer fortgeschrittenen Gruppe könnte zu diesem Thema improvisiert werden. Diese Möglichkeiten werden jedoch erst zu einem späteren Zeitpunkt der Sitzung aufgegriffen. 

\subsection*{Warming Up}
Bevor unsere Stimme als Instrument eingesetzt wird, ist es wichtig, sie "`aufzuwärmen"'. In diesem Teil werden zu Beginn Übungen angeleitet, welche zu einer tieferen Körper- und Atemwahrnehmung führen sollen (z.B. "`Körperwanderung"' und "`Atmen in unterschiedliche Rumpfbereiche fließen lassen"'). Insbesondere gezielte Atemübungen für die Vertiefung der Atmung bei COPD (wie die Lippenbremse oder auch Übungen, welche wie im Qi Gong mit Bewegungen verbunden werden) sowie Dehnungsübungen für den Körper sollen hier einbezogen werden. Dieser Teil ist sehr wichtig, um auf das Singen vorzubereiten. Je besser die Konzentration auf das Ausatmen gelenkt und eine entspannte Körperhaltung eingenommen wird, desto leichter und lustvoller wird die spätere Singerfahrung. 

Zudem ist gerade für COPD-Patienten eine umfassende Vorbereitung auf das Singen über Atem- und Körperübungen wichtig, um möglichen auftauchenden Ängsten z.B. vor Atemnot oder zu großer körperlicher Belastung entgegenzuwirken. 

Wie in der Singforschung gezeigt wurde, kann eine bewusste und geführte Ausatmung der oftmals unter COPD-Patienten vorherrschenden Schnappatmung (siehe Kapitel \ref{chapter:copd}) entgegenwirken und so die physische Belastung auch nachhaltig verbessern. Beim darauffolgenden Singen wird dies automatisch geübt, da beispielsweise ein Liedtext sich über eine längere melodische Phrase hinwegziehen kann und die musikalische Ausgestaltung das Einatmen an bestimmten Stellen fordert. Diese musikalischen Bögen würden durch ständiges Zwischenatmen unterbrochen werden, so dass der musikalisch fühlende und hörende Mensch vermutlich mit der Zeit den Ehrgeiz entwickelt, diese Bögen halten zu können. Natürlich soll im Rahmen dieses Singens jeder in seinem Maße atmen können. Wie in der neueren Singforschung jedoch bereits belegt werden konnte, gleicht sich die Atmung der Mitglieder einer Singgruppe im Verlauf immer mehr an. Dies kann zum einen ein  Gruppengefühl stiftendes Erlebnis darstellen. Zum anderen hat dies körperlich einen großen Effekt hinsichtlich einer ruhigeren und vertieften Atmung, wodurch Momente des Loslassens, der Entspannung entstehen können. In dieser Entspannung können nun auch durch den zuvor bestehenden Stress überlagerte Gefühle hervortreten und Raum bekommen \autocite[vgl.][59]{ehrmann2004}. 

Durch Atem- und Stimmübungen im hier angedachten Sinne können darüber hinaus noch weitere körperliche Effekte erzielt werden, welche bei einer COPD-Erkrankung wichtig sind: durch die stärkere Einbeziehung des Zwerchfells kann dessen Beweglichkeit verbessert werden, welche bei obstruktiven Lungenerkrankungen oftmals eingeschränkt ist. Zudem kann die Beweglichkeit des Brustkorbs gesteigert und die Atemhilfsmuskulatur gelockert werden, welche durch die veränderte Atmung meist verspannt ist. 

An die Atem- und Körperwahrnehmungsübungen schließen Übungen zum Aufwärmen der Stimme an, welche dem stimmbildnerischen Bereich entlehnt sind. \footnote{Durch meine Erfahrungen mit den "`Singenden Krankenhäuser"' habe ich einige sehr lustvolle und humorvolle Übungen für die Stimmarbeit mit Laiengruppen kennengelernt, welche zum Teil in "`Chanten: Eintauchen in die Welt des heilsamen Singens"' von Wolfgang Bossinger und Wolfgang Friederich (2013) nachzulesen sind.}

\subsection*{Musiktherapeutische Stimmarbeit}
Um den Einstieg in diesen "`stimmlichen Hauptteil"' zu erleichtern, kann beispielsweise ein ritualisiertes Anfangslied den Übergang gestalten. 
Während zu Beginn die Struktur durch komponierte Lieder und Stimmübungen noch mehr vorgegeben werden kann, um dem zuvor entwickelten "`Sicherheitsgedanken"' zu folgen, sollten im Verlauf des Therapieprozesses die stimmlichen Einheiten weniger gelenkt und zunehmend freier gestaltet werden. 

In diesem Teil können nun alle unter Kapitel \ref{methodische_moeglichkeiten_der_stimmarbeit} beschriebenen methodischen Möglichkeiten der Stimmarbeit angewandt werden. 
Hinsichtlich dem Therapieziel "`Steigerung der Selbstwirksamkeit"' können die Gruppenteilnehmer beispielsweise eigene Ideen für die Ausgestaltung von Vokalimprovisationen oder Situationslieder wie z.B. Regeln, Aufbau oder der Einbezug zusätzlicher Instrumente einbringen. Zudem ist es möglich, dass in diesen Teil auch rezeptive Angebote einfließen, wenn sich dies inhaltlich anbietet.

\subsection*{Abschluss}
Am Ende der Sitzung sollte noch genügend Raum und Zeit für eine "`Abschlussrunde"' eingeplant werden, um das zuvor Erlebte besser integrieren zu können und einen Übergang in den weiteren Tagesablauf zu schaffen. Hier besteht die Möglichkeit, die gesammelten Erfahrungen, entstandene Fragen und das aktuelle Befinden mitzuteilen und sich über diese Aspekte auszutauschen. Ein Ausblick auf die nächste Sitzung kann bereits einen Bogen zum nächsten Zusammentreffen spannen.

\section{Zusammenfassende Betrachtung}
Die vorherigen Ausführungen beschreiben "`eine"' mögliche Form für die musiktherapeutische Arbeit mit COPD-Patienten. Die Kombination aus Atem-, Körper- und Stimmübungen, Improvisation und Gespräch versucht den unterschiedlichen Anforderungen und Besonderheiten in körperlicher und psychischer Hinsicht gerecht zu werden, welche die Arbeit mit dieser Klientel an den Therapeuten stellt. Im folgenden, abschließenden Kapitel werden die vorherigen Ausführungen nochmals reflektiert und zusammengefasst sowie ein Ausblick auf mögliche, sich anschließende Schritte gegeben.




%: ----------------------- paths to graphics ------------------------
% change according to folder and file names
\ifpdf
    \graphicspath{{5_konzept/figures/PNG/}{5_konzept/figures/PDF/}{5_konzept/figures/}}
\else
    \graphicspath{{5_konzept/figures/EPS/}{5_konzept/figures/}}
\fi

\newpage\thispagestyle{empty}
% ---------------------------------------------------------------------------
%: ----------------------- end of thesis sub-document ------------------------
% ---------------------------------------------------------------------------
