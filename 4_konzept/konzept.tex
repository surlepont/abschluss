% this file is called up by thesis.tex
% content in this file will be fed into the main document

%: ----------------------- introduction file header -----------------------
\chapter{Konzeptionelle Ueberlegungen}
\label{chapter:konzeptionelle_ueberlegungen}
\setlength{\epigraphwidth}{7.0cm}
\epigraph{To study and not think is a waste. To think and not study is dangerous.}{Confucius}

\section{Psychodynamische Überlegungen}
\label{psychodynamische ueberlegungen}
In Kapitel 2 wurde bereits dargelegt, dass die Diagnose einer COPD auch auf psychischer Ebene für den Patienten mit Veränderungen verbunden sein kann. In welchem Ausmaß dies geschieht und mit welchen Folgen es verbunden sein kann, ist sehr individuell verschieden und hängt mit den jeweiligen biographischen Hintergründen und den damit verbundenen Entwicklungsmöglichkeiten hinsichtlich einer stabilen, gefestigten psychischen Verfassung sowie mit einem unterstützenden sozialen Umfeld zusammen. 
Wie bereits unter 2. näher erläutert, wird eine COPD nur unter bestimmten Umständen ausgebildet: genetische Veranlagung (ein sehr verschwindend geringer Teil), Umweltverschmutzung und Einatmen von Schadstoffen bspw. im beruflichen Umfeld sowie zum größten Teil durch einen länger andauernden Nikotin-Abusus (betrifft ca. 80\% der Patienten). 

Was bedeutet aber letzteres nun für die therapeutische Behandlung dieser Patienten? M.E. scheint es wichtig diesen Aspekt mitzudenken und aufzugreifen, da damit oftmals eine Sucht spezifische psychische Struktur verbunden ist, welche Besonderheiten insbesondere in der Gegenübertragung mit sich bringt und somit für die Behandlung abhängiger Patienten die Auseinandersetzung mit diesem Thema unumgänglich erscheint. Es ist mir natürlich bewusst, dass mit dieser Erklärung nicht alle potentiellen Patienten eingeschlossen sein werden, aber aufgrund der hohen Relevanz der Nikotinabhängigkeit in Verbindung mit der Ausbildung einer COPD bedarf es eines Einbezugs dieser Problematik in therapeutische Konzepte. Es gilt natürlich in der Praxis diese Annahmen stets zu überprüfen und bei Bedarf anzupassen bzw. individuell zu verändern. 

Bevor jedoch auf spezielle psychodynamische Aspekte eingegangen wird, bedarf es einer kurzen definitorischen Erläuterung, was unter dem Phänomen Sucht verstanden wird. 

todo Zigaretten-Abusus und Suchtstruktur - was bedeutet das? Eine Sucht wird allgemein folgendermaßen definiert todo: 

Sucht und Ich-Schwäche. In Krisen Rückgriff auf frühe Bewältigungsstrategien.. Bedeutung des Rauchens hinsichtlich einer Ersatzbefriedigung...

Es kann davon ausgegangen werden, dass bei Menschen mit einer ausgeprägten Suchtproblematik in der frühen Kindheit ein wichtiger Erfahrungsbereich nicht ausreichend genährt wurde: um ein Gefühl für eigene Bedürfnisse zu entwickeln sowie die Wichtigkeit des sich dafür Einsetzens zu erkennen und eigenverantwortlich umzusetzen, bedarf es der frühen Erfahrung, dass ein Kind im Außen diese auf sich bezogene Zuwendung und Wertschätzung erfahren hat. Über diesen Prozess ...

Bei Stern finden sich sehr essentielle Vorstellungen für die Entwicklung eines Selbstempfindens wieder. Diese sind jedoch nicht nur in Bezug auf die kindliche Entwicklung von großer Wichtigkeit, sondern bleiben es für das gesamte Leben. Stern unterscheidet hierbei zwischen sechs verschiedenen Bereichen, welche letztlich das Selbst des Menschen bilden: das auftauchende Selbst, das Kern-Selbst, das intersubjektive Selbst, das verbale Selbst sowie mittlerweile das narrative Selbst. Dabei gliedert er seit ein paar Jahren die Empfindung eines Kern-Selbst in zwei Bereiche: zum Einen in die Empfindung eines Kern-Selbst mit seinen drei Invarianzen von Urheberschaft, Kohärenz und Kontinuität und zum Anderen in die Empfindung eines Kern-Selbst in Gemeinschaft mit dem Anderen.  Wie in der Grafik ersichtlich wird, haben die ersten 3 Bereiche nach Sterns neueren Erkenntnissen ihren Ursprung bereits in der pränatalen Phase.

todo Abbildung 
Abb: Stern 2007: XX

Da an dieser Stelle nicht auf alle Bereiche des Selbstempfindens ausführlich eingegangen werden kann, empfiehlt sich ein Blick in Daniel Sterns Werk „Die Lebenserfahrung des Säuglings“, in welchem der Entwicklungspsychologe, Säuglingsforscher und Psychoanalytiker in sehr anschaulicher Weise die Entwicklung des Säuglings anhand der Entwicklung eines umfassenden Selbstempfindens erläutert.
In Bezug auf eine entwickelte COPD scheint es naheliegend, sich mit der Entwicklung eines Kern-Selbst und hier insbesondere mit der Invarianz der Urheberschaft zu beschäftigen, wobei nicht außer Acht gelassen werden darf, dass alle Bereiche ineinander greifen und sie letztlich nur theoretisch auseinanderdividiert werden können.

Nun aber warum scheint eine besondere Auseinandersetzung mit der o.g. Invarianz in Bezug auf eine COPD sinnvoll?!

Das Empfinden eines Kern-Selbst entsteht aus dem Zusammenspiel der zuvor genannten Invarianzen der Selbsterfahrung, d.h. unveränderlicher Erfahrungsmuster, die zu einer Organisation und Struktur innerer Prozesse führen. Wie aus der Grafik hervorgeht, handelt es sich um Entwicklungsbereiche, die sich hier natürlich auf den Lebensanfang beziehen, jedoch die Grundlage jeglicher Selbstempfindung bilden. So können sich die einzelnen Bereiche immer weiter ausdifferenzieren, die Basis jedoch bleibt bestehen. 
Stern beschreibt das „Kern-Selbst-Empfinden […] (als) ein erfahrungsgeleitetes Empfinden von Vorgängen, das wir normalerweise als völlig selbstverständlich voraussetzen und uns nicht bewusst machen. […] Das Selbstempfinden ist kein kognitives Konstrukt; es ist die Integration des Erlebens.“ \autocite[106f.]{stern2007} Er bezeichnet es sogar als „die Grundlage für alle differenzierten Selbstempfindungen, die sich später entwickeln werden“ \autocite[106f.]{stern2007}. 

Was passiert jedoch, wenn ein Mensch spürt, dass er zuvor selbstverständliche körperliche Prozesse, wie im Fall der COPD die Atmung, nicht mehr so steuern kann, wie er möchte? Dies spricht meines Erachtens in erster Linie die Invarianz der Urheberschaft an. Sie umfasst das Empfinden, Urheber eigener Handlungen zu sein als auch Nicht-Urheber der Handlungen anderer. Dies ist stets verbunden mit einem willentlichen Vorgehen, der propriozeptiven Wahrnehmung als auch dem Wissen, dass dieses Vorgehen bestimmte Konsequenzen nach sich zieht \autocite[vgl.][106, 114f.]{stern2007}. Im Falle der pneumologischen Veränderungen bei einer COPD, welche stets einhergehen mit ansteigender Atemnot scheint dieses Gefüge nun ins Wanken zu kommen: die Atmung als "selbstverständlicher" und meist nicht bewusst gesteuerter Vorgang verändert sich und dem COPD-Patienten scheint die Kontrolle über diesen Vorgang in manchen Situationen zu entgleiten. Dabei gerät jedoch primär der erste Teil dieser Invarianz, das willentliche Vorgehen, in eine unsichere Position, während die propriozeptive und im Falle der COPD auch die viszerozeptive Wahrnehmung intakt ist und eine Konsequenz für diese körperlichen Vorgänge erahnt werden kann. Dies kann verständlicherweise zu Unsicherheit führen.

Wiederholt sich dieser Vorgang stetig, kann es zu einer Schwächung regulierender Selbstobjekte führen, welche bereits im Säuglingsalter durch die Interaktion mit einem selbstregulierenden Anderen beginnen, sich herauszubilden \autocite[vgl.][338f.]{stern2007}. Je nach individueller Ausprägung der regulierenden Selbstobjekte kann dieser Vorgang früher oder später in eine Regression des Patienten münden. Nun wird die Regulierung auftauchender affektiver Zustände im Außen wieder wichtiger. 
An dieser Stelle kann an eine frühe Erfahrungswelt im Rahmen eines therapeutischen Settings angeknüpft werden, wenn ein geschützter, haltender, stützender und nährender Rahmen geschaffen werden kann \autocite[vgl.][8ff.]{decker-voigt2008}. Insbesondere die Arbeit mit der Stimme eignet sich hier besonders. Da die Stimme der primären Bezugspersonen am Anfang des Lebens i.d.R. verbunden wird mit der Erfahrung eines geschützten, nährenden Raums, "werden wir [lebenslang] in den Tiefen unseres Unbewussten mit Stimmausdruck eine >heile Welt< assoziieren" \autocite[282]{deckervoigt2000}. Das "heil" bezieht Decker-Voigt in diesem Zusammenhang darauf, dass uns der Klang der Stimme an eine Zeit erinnere, in der wir Kränkungen, Beängstigungen und Verletzungen seelisch noch ertragen konnten \autocite{deckervoigt2000}. Eine detaillierte Ausführung, in wie weit gerade die Arbeit mit Körper, Atem und Stimme zu einer Wiederentdeckung bzw. Nachentwicklung dieses Bereiches führen kann, ist unter Kapitel 3 nachlesbar.

Darüber hinaus scheint in Bezug auf eine angstfreie Ausbildung eines Kern-Selbst auch ein sicherer, haltender Rahmen notwendig. Greift man zurück auf die Erfahrungen des Säuglings, so wissen wir, dass die Entwicklung stets gekoppelt ist an die Verfügbarkeit der primären Bezugspersonen und ihrem Umgang mit dem Säugling. Für eine gelingende Entwicklung ist es wichtig, dass die primären Bezugspersonen (i.d.R. Mutter und Vater) feinfühlig auf das Kind eingehen, als sichere emotionale Basis für das Kind verfügbar sind (Begriffe aus der Bindungstheorie nach J. Bowlby und M. Ainsworth, siehe \cite{brisch2013} als auch durch Synchronisationsprozesse mit dem Säugling zur Ausbildung einer stabilen psychischen Struktur beitragen. Da zu Beginn des Lebens noch nicht die Möglichkeit zur Selbstregulation gegeben ist, sind auch für diesen Funktionsbereich die primären Bezugspersonen von großer Wichtigkeit. Wird der Säugling allein gelassen mit dieser Überstimulierung durch unbekannte Reize und kann sein eigenes Gefühlschaos nicht selbst regulieren, wird er sich vermutlich ängstlich zurückziehen. Hat er jedoch im Außen ein (markiert) spiegelndes \autocite[vgl.][153]{fonagy2004}, feinfühliges Gegenüber, so kann er nach und nach diese nun sich ausbildenden Repräsentanzen in seine psychische Struktur integrieren. 
Übertragen auf die Situation eines erwachsenen Menschen mit COPD kann dies bedeuten, dass er für die Bewältigung seiner gesundheitlichen Krise und zur Prävention vor komorbiden psychischen Störungen von einem regulierenden Anderen profitieren würde. Häufig jedoch sind die näheren Angehörigen aufgrund eigener Involviertheit nicht in der Lage, diesen stützenden Part zu übernehmen oder die Beziehung ist aufgrund der Krankheitssituation bereits zu sehr belastet. 

Daher kann es hilfreich sein, außerhalb der gewohnten sozialen Bezüge einen Raum für sich in Anspruch nehmen zu können, in dem es um die eigene Person geht, so wie sich zu Beginn des Lebens in einem geschützten Rahmen die Handlungen der Bezugspersonen am Säugling orientieren. Im Rahmen einer tiefenpsychologisch fundierten Musiktherapie, wie sie in dieser Arbeit vertreten wird, steht stets der "Musik erlebende und sich durch Musik ausdrückende Mensch als Klient" im "Zentrum der Aufmerksamkeit" \autocite[4]{timmermann2004}. Für den Umgang mit der Erkrankung bringt jeder vor dem Hintergrund seiner individuellen Kindheitsgeschichte Copingstrategien und Ressourcen mit, die ihm helfen, die Situation, zu bewerkstelligen. In manchen Fällen ist es jedoch sinnvoll, sich dieser gewahr zu werden, zu verstehen, wie und aus welcher Situation diese entstanden sind und evtl. neue Wege auszuprobieren. 

todo: Der Intermediäre Raum Winnicott


\section{Therapeutische Grundhaltung} 
(Timmermann) Setting: wenn in Reha - eher ressourcenorientiert, weniger Aufarbeitung als vielmehr Krankheitsbewältigung, anschließen an positive Erlebniszusammenhänge, welche Form des Singens-> beispielhafter Aufbau… da in Reha, Selbsthilfegruppen, Krankenhaus eher kein fester eigener Raum zur Verfügung steht, Mobilität gewährleisten… mit wenig Mitteln viel erreichen…
Sehr wichtig: Gegenübertragung wahrnehmen und nicht "ausagieren", Gefahr in Abhängigkeitsstruktur der Klienten hineingezogen zu werden.

Übungs- und Erlebniszentrierung im Spannungsfeld zwischen Körper- und Atemarbeit, Achtsamkeitslenkung, Gesangspädagogik und musiktherapeutischer Arbeit. Abwehr gegenüber psychotherapeutischer konfliktzentrierter Ausrichtung wegen negativer Konnotation

\section{Therapieziele}
Bewältigung der Krisensituation: Todesangst 

Ziel ist es einen Raum für Krankheitsbewältigung, Steigerung der Selbstwirksamkeit und Prävention angesichts häufig ausgebildeter Komorbiditäten wie Depression und Angststörungen sowie sozialer Isolation zu schaffen. Ressourcenorientiert und auf die Bedürfnisse der Patienten abgestimmt wird die Stunde gestaltet.

%  Musiktherapie mit Menschen in Lebenskrisen: Ziel musiktherapeutischer Arbeit ist es hier, die Möglichkeiten zum persönlichen Wachstum und zur Weiterentwicklung zu erkennen und zu nutzen, die in jeder Krise stecken. Hierzu ist es hilfreich, neue Sichtweisen auf sich selbst wie auf die persönlichen Lebensumstände zu entwickeln. Die Musik als ergänzendes Medium zur Sprache, der Wechsel vom Darüber-Reden zum Es-Spielen erweist sich als überaus hilfreich, um aus den herkömmlichen Denk- und Fühlschablonen auszusteigen. Gleichzeitig werden in den musikalischen Gestaltungen auch eigene Verwicklungen wie Potenziale deutlich, die im Gespräch im Schutze der therapeutischen Beziehung aufgearbeitet werden können.  http://www.klangwerk-musiktherapie.de/pages/t_erw12.html

Steigerung der Selbstwirksamkeit…
Steigerung der Lebensqualität
Prävention vor Ausbildung möglicher Komorbiditäten sowie sozialer Isolation
Krankheitsbewältigung

\section{Gruppentherapie oder Einzeltherapie}
Gedanken zur Zielgruppe
Gerade für die Krankheitsbewältigung kann hier der Austausch innerhalb einer diagnosespezifischen Gruppe hilfreich und unterstützend wirken. Um jedoch den oben entwickelten Gedanken in Bezug auf ein haltendes, stützendes und spiegelndes Gegenüber innerhalb eines geschützten Rahmens, in welchem Ausprobieren, Reflexion, Stärkung und Veränderung möglich werden, gerecht zu werden, bedarf es einer therapeutisch ausgebildeten Leitung dieser Gruppe. 

Scham
Schutz
Krankheitsbewältigung
Größe der Gruppe

\section{Einbezug von Körper und Atem} 
Da es sich bei der COPD primär um eine körperliche Erkrankung handelt, soll an dieser Stelle noch eine weitere Theorie, die des "Embodiments", hinzugezogen werden, die sich auf die Wechselwirkung von Körper und Psyche bezieht. Wie bereits zu Beginn dieser Arbeit darauf hingedeutet wurde, wird hier nicht der Zusammenhang einer Suchtstruktur mit der Ausbildung einer COPD untersucht; dieses würde vermutlich zu weiteren theoretischen Annahmen führen. Es handelt sich hierbei jedoch um eine Erkrankung, die sich vor dem Hintergrund unterschiedlicher Faktoren entwickeln kann (siehe Kapitel 2.3).
Vier Vertreter unterschiedlicher Disziplinen (Kognitionswissenschaften, Psychologie, Neurobiologie und Körperarbeit) haben zusammengetragen, was aus ihren unterschiedlichen Blickwinkeln zum Zusammenhang von Körper und Psyche wichtig erscheint. Entstanden ist diese Zusammenarbeit aus der gemeinsamen Erfahrung, dass der Zusammenhang von Psyche und Körper in vielen Bereichen noch mangelhaft Beachtung findet und gerade in therapeutischen und beratenden, aber auch wissenschaftlichen Arbeitsfeldern, in deren Fokus der Mensch steht, wichtiger Bestandteil der Betrachtung des Einzelnen sein sollte.
Storch, Cantieni, Hüther und Tschacher haben jedoch in ihrer Publikation nicht das Rad neu erfunden, sondern bestehendes Wissen und Ideen zum Thema zusammengetragen und weiterentwickelt. Embodiment-Theorien gehen davon aus, dass eine Wechselwirkung "zwischen allem, was als Körpergeschehen aufgefasst werden kann (dies beinhaltet einzelne motorische Aktionen und Bewegungsabläufe bis hin zu ganzen Verhaltenssequenzen) und dem psychischen System" \autocite[39]{hüther2010} besteht. todo

Körper- und Atemwahrnehmung sind jedoch nicht nur zur Unterstützung der Atemwegserkrankung sinnvoll und wichtig, sondern auch Achtsamkeit in Bezug auf die Sucht im Hier und Jetzt sein -> MBSR.

\section{Gedanken zu COPD und Singen}
Steigerung der Lebensqualität -> Singforschung

Wenngleich der Einsatz der (Sing-)Stimme insbesondere im Hinblick auf ein bewussteres und verlängertes (Aus-)Atmen m.E. in diesem Bereich sehr sinnvoll erscheint, so ist gleichzeitig auch Vorsicht geboten. Bei Patienten mit COPD kann es zu entzündlichen Vorgängen rund um den Stimmapparat aufgrund der medikamentösen Behandlung und einer geschwächten Immunabwehr kommen. In diesen Fällen ist es notwendig, durch einen Phoniater abklären zu lassen, ob die Stimme der Schonung bedarf oder aber der gezielte und bedachte Einsatz der Stimme zu einer Besserung der Stimmfähigkeit beitragen kann \autocite[vgl.][103ff.]{alavi2009}.

\section{Gedanken zum Setting}
Welche Einrichtungen kämen in Frage? Warum? 

\section{Indikation/ Kontraindikation}
Wie bereits unter \label{psychische_komorbiditaet} erläutert, manifestieren sich Angst- und Panikstörungen sowie Depression bereits in den frühen Stadien der Erkrankung und nehmen in der Regel mit dem Fortschreiten der Erkrankung nicht zu. Daher gilt es hier sich nicht nach den Schweregraden, sondern nach dem Bedarf an psychotherapeutischer Begleitbehandlung zu orientieren. 

Generell gilt jedoch aufgrund des erhöhten Risikos zur Ausbildung einer psychischen Komorbidität für jeden Patienten, der daran Interesse hat und davon profitieren könnte, eine Möglichkeit zur Teilnahme an diesem therapeutischen Behandlungsangebot bereitzuhalten, da ein wichtiges therapeutisches Ziel in der Prävention zur Ausbildung einer psychischen Erkrankung als auch der Krankheitsbewältigung liegt.

Wie bei jeder gruppentherapeutischen Behandlung gelten auch hier folgende Ausschlusskriterien: akute Psychosen/ hirnorganische Störungen, Schwierigkeiten, einen Leiter zu akzeptieren und/oder interpersonelle Beeinträchtigungen. Darüber hinaus gilt es die Motivation hinsichtlich der Behandlung zu überprüfen. Weitere mögliche Kriterien zur Gestaltung einer Gruppe im psychotherapeutischen Setting können bei Strauß und Mattke gefunden werden \autocite[vgl.][78-88]{mattke2007}.

Eine weitere Kontraindikation könnte fehlende Mobilität bedeuten, da das Gruppenangebot selbstständig oder mit Hilfe aufgesucht werden muss. Bei gleichzeitigem Bedarf an psychotherapeutischer Begleitung sollte hier ein therapeutisches Angebot im aufsuchenden Einzelsetting in Betracht gezogen werden.

Sollte es im Vorgespräch mit einem Patienten Anzeichen für eine Traumatisierung geben, gilt es hier abzuwägen, ob eine Gruppenbehandlung im hier angedachten Sinne angemessen erscheint. Oftmals ist es gerade für diese Patientengruppe wichtig, in einem geschützten und sehr strukturierten Rahmen äußere und innere Sicherheit aufzubauen, um sich dann evtl. freieren Formen anzunähern. Da bei diesen Menschen das "Erstarren und Verstummen [...] einen überlebensnotwendigen Schutz darstellen" \autocite[68]{rittner2012} und Musik einen Trigger für Retraumatisierungen darstellen kann, gilt es hier respektvoll und sehr behutsam vorzugehen.

\section{Praktische Umsetzung}
In einem geschützten Raum wird die Möglichkeit geschaffen, in Kontakt zu gehen, sich auszutauschen eigenes Erleben zu teilen und zu reflektieren. In der Gruppe Rückhalt zu finden, sich nicht mehr isoliert und alleine mit der Erkrankung zu fühlen, dies jedoch in einem therapeutischen Rahmen, welcher die Möglichkeit für die Hinterfragung der eigenen Lebensbehandlung sowie die Chance zur Veränderung bietet, sind die Intention dieses Konzepts. Abgestimmt auf das Krankheitsbild soll jedoch nicht nur auf der kognitiven, Gesprächsebene angesetzt werden, sondern über Körper- und Atemübungen bis hin zum stimmlichen Ausdruck die eigene Körpersensibilität und der Zugang zur eigenen emotionalen Verfassung und ihrem Ausdruck ausgeweitet werden, um dadurch wieder zu mehr Sicherheit und Selbstvertrauen in die eigenen selbstwirksamen Kräfte zu gelangen. Wenn die institutionellen und finanziellen Rahmenbedingung es ermöglichen, wäre ein Grundinstrumentarium hilfreich, insbesondere in der Anfangsphase, jedoch nicht notwendig. Der Einstieg über die instrumentale Improvisation kann hinweghelfen über die Hemmung des Stimmeinsatzes (wie weiter zuvor bereits erläutert).  

Nachdem unter \label{psychodynamische ueberlegungen} einige psychodynamische Phänomene bei einer Suchtproblematik zusammengetragen wurden, sollen diese nun auch bei den Überlegungen zum Aufbau und Inhalt der Sitzungen Berücksichtigung finden. 



Geben und nehmen stein, sich verbinden

mit kleinen Einheiten beginnen, um Abwehr zu reduzieren sicherheit Struktur. evtl mit summen einsteigen, feine Vibration, ich bin da, hörbar, aber zeig noch nicht so viel von mir


\subsection{Überlegungen zum Aufbau der Sitzungen}

Eine Sitzung sollte jedoch 60 Minuten nicht überschreiten, um die körperliche Belastung der Patienten nicht zu strapazieren.

Wichtigkeit des Rituals: Ritual als sicheres Kontinuum vor dem Hintergrund auftretender Atemnot, plötzlicher Krankheitsverschlechterung. 
Vom Strukturierten zu immer Freieren Stimmäußerungen
Mischung aus Atem- Körperwahrnehmung, Tönen, Liedern, Stimmimprovisationen evtl. unter Hinzunahme anderer Instrumente…(wäre sicherlich für bestimmte Situationen hilfreich, jedoch nicht in jedem Rahmen umsetzbar). 
Aufmerksamkeitslenkung. 

Ankommen: 
Begrüßung, Frage in die Runde, wie jeder einzelne heute hier ist, ob er ein bestimmtes Thema mitgebracht hat, ihn etwas bewegt, was er teilen möchte etc. evtl. herausarbeiten eines gemeinsamen Themas, worüber zu einem späteren Zeitpunkt improvisiert werden, in einer geleiteten Imagination oder einem thematisch passenden Lied dieses aufgegriffen werden kann.

Introspektion: 
Körperwahrnehmung, Atemwahrnehmung, gezielte Atemübung, die eigene Stimme zum Klingen bringen. Dieser Teil ist sehr wichtig, um auf das Singen vorzubereiten. Je besser die Konzentration auf den Ausatem und eine entspannte Körperhaltung, desto leichter und lustvoller wird die spätere Singerfahrung (siehe Kapitel 2). Wie in der Singforschung gezeigt werden konnte, kann eine bewusstere und geführte Ausatmung der oftmals unter COPD-Patienten vorherrschenden Schnappatmung (siehe Kapitel zu COPD) entgegenwirken und so die physische Belastung auch nachhaltig verbessern. Beim darauffolgenden Singen wird dies automatisch geübt, da beispielsweise ein Liedtext sich über eine längere melodische Phrase hinwegziehen kann und die musikalische Ausgestaltung das Einatmen an bestimmten Stellen fordert. Diese musikalischen Bögen würden durch ständiges Zwischenatmen unterbrochen, so dass der musikalisch fühlende und hörende Mensch vermutlich mit der Zeit den Ehrgeiz entwickelt, diese Bögen halten zu können. Natürlich soll im Rahmen dieses Singens jeder in seinem Maße Atmen können, nur wie in der neueren Singforschung bereits belegt werden konnte, gleicht sich die Atmung der Mitglieder einer Singgruppe im Verlauf immer mehr an. Was zum einen körperlich einen großen Effekt hinsichtlich einer ruhigeren, entspannteren Atmung als auch sozial ein Gruppengefühl stiftendes Erlebnis darstellen kann.

In Kontakt gehen: 
Interaktive, auflockernde Übungen (Löwe, Bahn etc.), ritualisiertes Anfangslied, Mantren evtl. gewünschte Lieder die Ausdruck für bestimmte Wünsche, Aussagen des Betreffenden, Hoffnungen, Ängste etc. sein können
(Stärkung der Selbstwirksamkeit: Vokalimprovisation, eigene Ideen für die Ausgestaltung wie z.B. Regeln, Aufbau, zusätzliche Instrumente) Dieser Teil sollte sanft eingeführt werden. Nicht gleich zu Beginn, da es oftmals Überwindung kostet und vor allem Erfahrung mit der eigenen Singstimme und somit Sicherheit mit dergleichen bedarf.

Rezeptiv: 
Besungen/ bespielt werden, die Erfahrung des umsorgt seins ansprechend

Abschluss: 
zurückkehren in die Runde. Austausch der Erfahrungen, entstandener Fragen, des aktuellen Befindens. Ausblick, Evtl. Abschlusslied


\subsection{Beispiele für Körper-, Atem-, und Stimmübungen sowie methodische Möglichkeiten der Stimmarbeit}



%: ----------------------- paths to graphics ------------------------
% change according to folder and file names
\ifpdf
    \graphicspath{{5_konzept/figures/PNG/}{5_konzept/figures/PDF/}{5_konzept/figures/}}
\else
    \graphicspath{{5_konzept/figures/EPS/}{5_konzept/figures/}}
\fi

\lettrine{T}{he} algorithm for

\newpage\thispagestyle{empty}
% ---------------------------------------------------------------------------
%: ----------------------- end of thesis sub-document ------------------------
% ---------------------------------------------------------------------------
