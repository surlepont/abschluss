%: ----------------------- introduction file header -----------------------
\chapter{Einleitung}
%\epigraph{Pommes sind lecker.}{Ben}
% the code below specifies where the figures are stored
\ifpdf
    \graphicspath{{1_introduction/figures/PNG/}{1_einleitung/figures/PDF/}{1_einleitung/figures/}}
\else
    \graphicspath{{1_einleitung/figures/EPS/}{1_einleitung/figures/}}
\fi

% ----------------------------------------------------------------------
%: ----------------------- introduction content -----------------------
% ----------------------------------------------------------------------

Musiktherapie wie sie im Folgenden beschrieben wird, gilt hier als ein psychotherapeutisches Angebot.
Thema Sucht und COPD wird in dieser Arbeit in Kapitel 4.1 gestreift. 
While robots ...

\begin{quote}
``It is not that difficult to build computers capable of playing chess or doing sums. Computers find it easy to do what we learned at school. But computers have a very hard time learning what children learn \textit{before} they start school: [...] navigating a backyard, recognizing a face; seeing.'' 
\end{quote}

\begin{itemize}
\item der erste Punkt
\item der zweite Punkt
\end{itemize}

\begin{enumerate}
\item der erste Punkt
\item der zweite Punkt
\end{enumerate}

\section{Motivation} % section headings are printed smaller than chapter names
\lettrine{H}{uman} beings have been dreaming to create intelligent \emph{robots} 

\subsection{Untergedängs}
tüdlötß
\subsubsection{unteruntergedaengs}
hehe

\paragraph{hahaha parapgrah}

\section{Outline}
\indent{The rest of this thesis is organized as follows: }

Chapter \ref{chapter:soa} 

\section{Research questions and contributions}
\label{sec:introduction:contributions}

\newpage\thispagestyle{empty}