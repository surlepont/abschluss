%: ----------------------- introduction file header -----------------------
\chapter{Einleitung}
\label{einleitung}
%\epigraph{Singen ist die Sprache der Wahrheit}{Sting}
% the code below specifies where the figures are stored
\ifpdf
    \graphicspath{{1_introduction/figures/PNG/}{1_einleitung/figures/PDF/}{1_einleitung/figures/}}
\else
    \graphicspath{{1_einleitung/figures/EPS/}{1_einleitung/figures/}}
\fi

% ----------------------------------------------------------------------
%: ----------------------- introduction content -----------------------
% ----------------------------------------------------------------------
Todo: noch nicht fertig!

In den letzten Jahrzehnten ist die Zahl der an Chronisch obstruktiver Lungenerkrankung (COPD= chronic obstructive pulmonary disease) leidenden Patienten rapide angestiegen - mittlerweile sind bereits rund 340 Millionen Menschen weltweit an COPD erkrankt. Sie steht momentan noch an vierter Stelle der häufigsten Todesursachen, laut der WHO soll sie bis 2030 sogar auf die dritte Stelle aufsteigen. Es handelt sich hierbei um eine nicht vollständig reversible und progrediente Lungenerkrankung, bei welcher die Patienten unter einer Obstruktion der Atemwege durch eine chronische Entzündung leiden. Dies geht einher mit häufiger Atemnot und starkem Auswurf.
Es wurde jedoch noch kein Medikament gefunden, dass die Erkrankung heilen oder wenigstens das Fortschreiten aufhalten kann. Dies mag u.a. der Tatsache geschuldet sein, dass die Forschung auf diesem Gebiet noch vergleichsweise jung ist. Aktuelle Forschungsergebnisse lassen jedoch auf eine Verbesserung der medikamentösen Therapie hoffen.

In der Behandlung kommen derzeit in erster Linie Pharmazeutika sowie Raucherentwöhnung, Atemtherapie, Bewegungstherapie und Entspannungstechniken wie z.B. autogenes Training zum Einsatz. In einigen Rehabilitationskliniken wird zudem psychologische Beratung angeboten, welche Patienten bei Bedarf in Anspruch nehmen können.

Viele Patienten sind leider nach wie vor nur sehr mangelhaft über ihre Erkrankung aufgeklärt und nehmen nur selten präventive therapeutische Angebote in Anspruch. Daraus resultiert jedoch oftmals ein Verhalten, dass nicht nur zur Verschlechterung der Erkrankung sondern auch zur Ausbildung von Komorbiditäten wie Herz-Kreislauf-Erkrankungen aber auch zu Depression und Angststörungen führt. Ein wichtiger Aspekt ist hierbei die Vermeidung von symptomverstärkenden Tätigkeiten (s.g. „Fear Avoidance“). Der Lungeninformationsdienst  berichtet, dass ca. 40\% der COPD-Patienten unter einer depressiven Symptomatik leide und fordert auf seiner Webseite, dass „eine psychotherapeutische Behandlung […] deshalb ein fester Bestandteil der COPD Therapie sein“ sollte. Auch an anderen Stellen, insbesondere bei Fachärzten, scheint dieser Aspekt stärker ins Bewusstsein zu rücken. Schaut man sich jedoch beispielsweise auf den Internetseiten der führenden pneumologischen (Reha-)Kliniken um, ist die psychologische Betreuung noch recht mangelhaft und die Krankheitsbewältigung i.S. einer psychotherapeutischen Begleitung noch kaum in die Behandlungskonzepte eingebettet. 

Ein weiterer interessanter Bereich in Bezug auf die nicht-medikamentöse Begleitbehandlung findet sich meines Erachtens in der neueren Singforschung. In unterschiedlichen Studien wurde der Effekt von Singen auf die Lebensqualität und den Krankheitsfortschritt der Patienten untersucht. Dabei zeigte sich vor allem, dass sich die Lebensqualität signifikant steigerte, insbesondere durch die positiven Erfahrungen, in Gemeinschaft zu sein. Meine Vermutung ist zudem, dass eine Steigerung der Lebensqualität in diesem Fall auch eng verbunden ist mit der verstärkten Aktivierung der Selbstwirksamkeit durch die Erfahrung, dass die eigene Stimme trotz der Schwere der Erkrankung durch die eigene Kraft zum Klingen gebracht werden kann und die Atmung wieder mehr in der eigenen Kontrolle liegt. So wird das Gefühl des „Ausgeliefertseins“ sukzessive verringert werden.
Diese beiden Aspekte, der Bedarf an psychotherapeutischer Begleitung u.a. für die Krankheitsbewältigung sowie zur Prävention vor Komorbiditäten als auch die positiven Wirkungen der Arbeit mit der Stimme, insbesondere dem Singen, möchte ich in dieser Masterarbeit konzeptionell zusammenbringen.



\newpage\thispagestyle{empty}