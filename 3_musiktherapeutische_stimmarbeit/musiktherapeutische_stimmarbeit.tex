% this file is called up by thesis.tex
% content in this file will be fed into the main document

%: ----------------------- introduction file header -----------------------
\chapter{Musiktherapeutische Stimmarbeit}
\label{chapter:musiktherapeutische_stimmarbeit}
\setlength{\epigraphwidth}{8.0cm}
\epigraph{Success depends upon previous preparation, and without such preparation there is sure to be failure.}{Confucius}
\ifpdf
    \graphicspath{{3_musiktherapeutische_stimmarbeit/figures/PNG/}{3_musiktherapeutische_stimmarbeit/figures/PDF/}{3_musiktherapeutische_stimmarbeit/figures/}}
\else
    \graphicspath{{3_musiktherapeutische_stimmarbeit/figures/EPS/}{3_musiktherapeutische_stimmarbeit/figures/}}
\fi
% ----------------------------------------------------------------------
%: ----------------------- introduction content -----------------------
% ----------------------------------------------------------------------
\lettrine{I}{n} this chapter, we introduce 


Herr \autocite[vgl.][12]{wolf2012} sagt, singen ist schön \autocite[9]{wolf2012}.
...them with 10Hz GNSS PVT\footnote{position, velocity, time} solutions. Because the MEMS-grade IMU exhibits a drift of more tha


\section{Einsatz der Singstimme in der Musiktherapie}
Bevor auf das Singen in der MT im Speziellen eingegangen wird, sollte der Blick auf den Gebrauch der Singstimme in der Gesellschaft gewendet werden, um ein klareres Gesamtbild dieses Themas entstehen lassen zu können.

Während das Singen, man beachte unseren großen Umfang an deutschem Liedgut, in unserer Großelterngeneration noch weit verbreitet war, so hat die nationalsozialistisch geprägte Zeit der 30er und 40er Jahre einen Keil in diesen selbstverständlichen Gebrauch der Singstimme in Gemeinschaft getrieben.
Die Verschandelung deutschen Liedguts mit nationalsozialistischem Gedankengut und den Einsatz desgleichen für propagandistische Zwecke sowie im Besonderen in der Jugendmusikbewegung der damaligen Zeit führte u.a. zu "der späteren Voreingenommenheit gegenüber gemeinsamem Singen im Allgemeinen und gegenüber dem deutschen Volkslied im Speziellen" \autocite[9]{wolf2012}.

Hinzu kommt die Entwicklung neuer Medien und die Möglichkeit eines geöffneten Zugangs zu diesen. Laut Wolf hat dies großen Einfluss auf "die Entwicklung hin zu einer Vereinzelung der Menschen und hin zu einer Veränderung des menschlichen Alltagverhaltens" \autocite[10]{wolf2012}, wodurch die Notwendigkeit der gemeinschaftlichen Freizeitgestaltung, welche in früheren Zeiten oftmals das gemeinsame Singen beinhaltete, nachließ. Nur an vereinzelten Schauplätzen wie z.B. im Gottesdienst, im Stadion, bei den Pfadfindern oder in Chören wird das gemeinschaftliche Singen und/oder Grölen noch aktiv praktiziert.

In der MT war der Einsatz der Singstimme ebenfalls lange Zeit negativ konnotiert und wurde "[\ldots] als "Konflikt vermeidende Technik" in den heilpädagogisch orientierten Bereich der Kindermusiktherapie oder das palliativ orientierte Feld der Geriatrie verortet" \autocite[10]{wolf2012}. So war das Singen von Liedern im Rahmen einer "psychotherapeutisch orientierten MT" lange Zeit ausgeklammert. Zudem wurde die Stimme zum klanglichen Ausdruck kaum genutzt, da man einen zu großen Widerstand seitens der Patienten erwartete. 

%Todo: Widerstand und Stimmeinsatz

Durch den sich in den letzten Jahren vollziehenden Paradigmenwechsel in der psychotherapeutischen Behandlung im Allgemeinen und der musiktherapeutischen im Speziellen von einem eher konfliktzentrierten Ansatz und dem Konzept der Katharsis hin zu einem eher ressourcenorientierten und stabilisierenden Arbeiten veränderte sich auch die Einstellung gegenüber dem Singen. Einer kleinen Forschergruppe aus Musiktherapeuten und -pädagogen, aber auch bekannten Ärzten (wie z.B. Dr. Gerald Hüther, Dr. Manfred Spitzer u.a.) ist es zu verdanken, dass wir heute mehr über die Wirkung des Singens auf Körper, Geist und Seele wissen. Sie waren es, die zu einer Etablierung von Singgruppenarbeit primär beigetragen haben. Im klinischen Bereich hat sich insbesondere Wolfgang Bossinger verdient gemacht und eine mittlerweile über Deutschland verbreitete Initiative, "Singende Krankenhäuser", ins Leben gerufen. Aber auch Sabine Rittner und Karl Adamek haben zu einem großen Erkenntnisgewinn und möglichen Vorgehensweisen in diesem Bereich verholfen.

Heute wird die (Sing-)Stimme im psychotherapeutischen Kontext in unterschiedlicher Art und Weise eingesetzt, genutzt und betrachtet. Sabine Rittner todo hat diese Bereiche kategorisiert, um ihre Komplexität zu entzerren. Diese insgesamt acht Kategorien fließen in die folgenden Ausführungen ein: 
Stimme als\ldots
\begin{enumerate}
\item Medium der verbalen und nonverbalen Beziehungsgestaltung
\item Methode in der körperorientierten Musikpsychotherapie
\item Diagnostikum im therapeutischen Gespräch
\item Indikator für die therapeutische Übertragung- und Gegenübertragung
\item Symptom
\item Ausdrucksmittel
\item Selbstheilungs-Mittel
\item Medium zur Tranceinduktion
\end{enumerate}
Im Rahmen der Diagnostik gibt die Stimme des Patienten bereits wichtige Hinweise auf dessen aktuelle Befindlichkeit, seine Stimmung und durch aufmerksames Hinhören kann erspürt werden, ob das Gesagte in sich "stimmig", sprich kongruent ist. Zudem gibt sie bereits Auskunft über den biografischen Hintergrund der sich äußernden Person, denn sie hat sich mit unseren über die Zeit gesammelten Lebenserfahrungen weiterentwickelt und diese in sich aufgesogen. So gilt die Stimme unter Experten als "Klingendes Hologramm der Persönlichkeit" \autocite{adamek1999} "lauthafte Biographie" \autocite{gundermann1994} und bei Rittner erfahren wir über den Klang der Stimme etwas über die "leib-seelische Gewordenheit" \autocite[211]{rittner2008} des Menschen.
Dabei scheint jedoch nicht so sehr der Inhalt des Gesagten aufschlussreich, sondern vielmehr der "Klang der Stimme (Klangspektrum, Modulation, Lautstärkeänderungen, Stimmsitz, Vokaleinsatz etc.), die Sprechweise (Artikulation, Phrasierung, Pausensetzung etc.) und die Atmung (Atemfrequenz, Atemtiefe, hörbares Einatmen, Sitz des Atemraumes im Körper etc.)" als auch "die Art der Gestaltung von Sprechpausen, Abbrüchen, Momenten des Innehaltens, Verzögerungen u.ä." \autocite[210]{rittner2008}. Diese stillen Momente, wie


Die Wirkung des Singens auf Körper und Psyche
Bossinger, Adamek und Cramer

Wirkungsfähigkeiten des Singens im therapeutischen Kontext

Die Rolle der Stimme für die Diagnostik




\newpage\thispagestyle{empty}
% ----------------------------------------------------------------------