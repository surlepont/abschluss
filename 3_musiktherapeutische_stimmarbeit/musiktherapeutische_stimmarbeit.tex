% this file is called up by thesis.tex
% content in this file will be fed into the main document

%: ----------------------- introduction file header -----------------------
\chapter{Musiktherapeutische Stimmarbeit}
\label{chapter:musiktherapeutische_stimmarbeit}
\setlength{\epigraphwidth}{8.0cm}
\epigraph{Success depends upon previous preparation, and without such preparation there is sure to be failure.}{Confucius}
\ifpdf
    \graphicspath{{3_musiktherapeutische_stimmarbeit/figures/PNG/}{3_musiktherapeutische_stimmarbeit/figures/PDF/}{3_musiktherapeutische_stimmarbeit/figures/}}
\else
    \graphicspath{{3_musiktherapeutische_stimmarbeit/figures/EPS/}{3_musiktherapeutische_stimmarbeit/figures/}}
\fi
% ----------------------------------------------------------------------
%: ----------------------- introduction content -----------------------
% ----------------------------------------------------------------------
\lettrine{I}{n} this chapter, we introduce 


Herr \autocite[vgl.][12]{wolf2012} sagt, singen ist schön \autocite[9]{wolf2012}.
...them with 10Hz GNSS PVT\footnote{position, velocity, time} solutions. Because the MEMS-grade IMU exhibits a drift of more tha

Bevor im Folgenden selbstverständlich mit dem Begriff der "Stimme" umgegangen wird, scheint es sinnvoll eine kleine Einführung in die Physiologie der Stimme zu geben, um zum Einen die Komplexität der Funktionsweise unseres körpereigenen Instruments zu verstehen und um zum Anderen eine gemeinsame Grundlage für die Ausführungen in den nun folgenden Kapiteln zu schaffen. Stimmphysiologische Hintergründetodo

In den folgenden Kapiteln wird 

\section{Geschichtliche Einführung in das Thema}
Zu Beginn der weiteren Ausführungen in diesem Kapitel sollte der Blick auf den Gebrauch der Singstimme in der Gesellschaft gewendet werden, um ein klareres Gesamtbild dieses Themas entstehen lassen zu können.

Während das Singen in unserer Großelterngeneration noch weit verbreitet war, man beachte unseren großen Umfang an deutschem Liedgut, so hat die nationalsozialistisch geprägte Zeit der 30er und 40er Jahre einen Keil in diesen selbstverständlichen Gebrauch der Singstimme in Gemeinschaft getrieben.
Die Verschandelung deutschen Liedguts mit nationalsozialistischem Gedankengut und den Einsatz desgleichen für propagandistische Zwecke, sowie im Besonderen in der Jugendmusikbewegung der damaligen Zeit geschehen, führte u.a. zu "der späteren Voreingenommenheit gegenüber gemeinsamem Singen im Allgemeinen und gegenüber dem deutschen Volkslied im Speziellen" \autocite[9]{wolf2012}.

Hinzu kommt seitdem die Entwicklung neuer Medien und die Möglichkeit eines geöffneten Zugangs zu diesen. Laut Wolf hat dies großen Einfluss auf "die Entwicklung hin zu einer Vereinzelung der Menschen und hin zu einer Veränderung des menschlichen Alltagverhaltens" \autocite[10]{wolf2012}, wodurch die Notwendigkeit der gemeinschaftlichen Freizeitgestaltung, welche in früheren Zeiten oftmals das gemeinsame Singen beinhaltete, nachließ. Nur an vereinzelten Schauplätzen wie z.B. im Gottesdienst, im Stadion, bei den Pfadfindern oder in Chören wird das gemeinschaftliche Singen und/oder Grölen noch aktiv praktiziert.

In der MT war der Einsatz der Singstimme ebenfalls lange Zeit negativ konnotiert und wurde "[\ldots] als "Konflikt vermeidende Technik" in den heilpädagogisch orientierten Bereich der Kindermusiktherapie oder das palliativ orientierte Feld der Geriatrie verortet" \autocite[10]{wolf2012}. So war das Singen von Liedern im Rahmen einer "psychotherapeutisch orientierten MT" lange Zeit ausgeklammert. Zudem wurde die Stimme zum klanglichen Ausdruck kaum genutzt, da man einen zu großen Widerstand seitens der Patienten erwartete (siehe weiter unten in diesem Kapitel). 

Durch den sich in den letzten Jahren vollziehenden Paradigmenwechsel in der psychotherapeutischen Behandlung im Allgemeinen und der musiktherapeutischen im Speziellen von einem eher konfliktzentrierten Ansatz und dem Konzept der Katharsis hin zu einem eher ressourcenorientierten und stabilisierenden Arbeiten veränderte sich auch die Einstellung gegenüber dem Singen\autocite[vgl.][11]{wolf2012}. Einer kleinen Forschergruppe aus Musiktherapeuten und -pädagogen, aber auch bekannten Ärzten (wie z.B. Dr. Gerald Hüther, Dr. Manfred Spitzer u.a.) ist es zu verdanken, dass wir heute mehr über die Wirkung des Singens auf Körper, Geist und Seele wissen. Sie waren es, die zu einer Etablierung von Singgruppenarbeit primär beigetragen haben. Im klinischen Bereich hat sich insbesondere der Musiktherapeut Wolfgang Bossinger verdient gemacht und eine mittlerweile über Deutschland verbreitete Initiative, "Singende Krankenhäuser", ins Leben gerufen. Aber auch Sabine Rittner und Karl Adamek haben zu einem großen Erkenntnisgewinn und möglichen Vorgehensweisen in diesem Bereich verholfen.

\section{Einsatz der (Sing-)Stimme im musiktherapeutischen Kontext}

Heute wird die (Sing-)Stimme im psychotherapeutischen Kontext in unterschiedlicher Art und Weise eingesetzt, genutzt und betrachtet. Sabine Rittner \autocite[vgl.][204 ff.]{rittner2008} hat diese Bereiche kategorisiert, um ihre Komplexität zu entzerren. Diese insgesamt acht Kategorien fließen zum Teil in die folgenden Ausführungen ein: 
Stimme als\ldots
\begin{enumerate}
\item Medium der verbalen und nonverbalen Beziehungsgestaltung
\item Methode in der körperorientierten Musikpsychotherapie
\item Diagnostikum im therapeutischen Gespräch
\item Indikator für die therapeutische Übertragung- und Gegenübertragung
\item Symptom
\item Ausdrucksmittel
\item Selbstheilungs-Mittel
\item Medium zur Tranceinduktion
\end{enumerate}

Bereits im Mutterleib beginnt der Beziehungsaufbau zwischen Mutter und Kind über stimmliche Äußerung. Ab der 23. Schwangerschaftswoche kann der Fötus seine Umwelt auditiv wahrnehmen und auf sie reagieren. Die Stimme der Mutter tritt dabei in den Vordergrund. Wie man durch Untersuchungen kurz nach der Geburt festgestellt hat, können Säuglinge die Stimme ihrer Mutter von der anderer Personen unterscheiden \autocite [vgl.][22f]{noecker-ribeaupierre2004}. 

Die stimmliche Äußerung ist für die Entwicklung des Kindes von besonderer Bedeutung: durch Abstimmungsprozesse mit den primären Bezugspersonen auf unterschiedlichen sensorischen Ebenen, insbesondere durch stimmliche Interaktion, lernt es, die Welt und sich über die s.g. amodale Wahrnehmung zu begreifen. Gerade in dieser Zeit wird deutlich, welche Bedeutung die Stimme im Hinblick auf die Mitteilung der eigenen emotionalen Gestimmtheit hat, wie sich Beziehung ohne Worte aber dennoch durch stimmlichen Ausdruck gestalten lässt. Insbesondere durch den Einsatz von Vokalen kann diese "emotionale Tönung" \autocite[205]{rittner2008} deutlich werden, wie weiter unten näher erläutert wird.

Rittner \autocite [vgl.][106f.]{rittner1990} betont zudem die therapeutische Relevanz der Polaritäten, die sich im Schreien und Lallen des Säuglings äußern. Mit dem ersten Schrei tritt der Säugling zum ersten Mal geräuschvoll mit seiner neuen/ alten Umwelt in Kontakt und entlädt dabei die zuvor im ersten Einatem aufgebaute Anspannung. Diese Form der Kontaktaufnahme mit der Außenwelt differenziere sich, so Rittner, in den darauffolgenden Wochen immer mehr zu einem "gerichteten Appell" in Bezug auf die Befriedigung der Grundbedürfnisse. Im Lallen hingegen zeige der Säugling aus einer befriedigten Grundstimmung heraus eine "lustbetonte, nach innen gerichtete, autoerotisch-regressive Handlung", die oftmals unterbrochen wird, sobald beispielsweise im Außen Geräusche auftreten oder eine Person das Zimmer betritt und ihn so aus seiner Versunkenheit herauszieht. 
Die o.g. Polaritäten können aus diesen beiden Handlungen heraus folgendermaßen beschrieben werden: "die Verbindung zwischen Innenraum und Außenraum, zwischen Regression, dem wohligen Sich-Einhüllen, und Progression, in der sich lebenserhaltende aggressive Anteile artikulieren" \autocite[vgl.][106f.]{rittner1990}.

Diese Aspekte sind wesentlich für die Gestaltung und die Arbeit im Rahmen einer therapeutischen Beziehung. Mithilfe musiktherapeutischer Stimmarbeit ist es uns somit möglich, in sehr direkter Form an diese sehr frühen Erfahrungswelten anzuknüpfen und an einer Ausbalancierung der o.g. Polaritäten zu arbeiten. 

Allerdings gilt es dabei u.a. zu beachten, dass der Mensch im Laufe seiner Entwicklung Hemmungsmechanismen ausbildet, welche auf die Sauberkeitserziehung sowie auf Sozialisationsprozesse zurückzuführen sind. Diese Mechanismen lassen sich gliedern in "Scham- und Peinlichkeitsgefühle als auch [in] psychische Instanzen, die das Gewissen und Schuldgefühle auslösen" \autocite [106f.]{rittner1990}. 

Durch das Singen kommen wir, wie bereits zuvor dargelegt, schnell in Kontakt mit unserer primär unbewussten Gefühlswelt. Aufgrund der Nähe zu "frühkindlichen Gefühlszuständen vor der Schamhemmung" (todo Klausmeier zitiert in Rittner 1990 todo) und damit einem sehr intensiven emotionalen Erleben kann der unvorbereitete Einsatz der Singstimme überflutend und angstauslösend hinsichtlich des ungewollten Zeigens von u.a. abgewehrten Selbstanteilen wirken. 

Da das Thema Scham m.E. wichtig ist für die Arbeit mit der Stimme im musiktherapeutischen Rahmen und bereits am Anfang dieses Kapitels in Wolfs Aussage zu Widerständen in diesem Kontext anklingt, soll an dieser Stelle darauf eingegangen werden.

\subsection{Scham und Stimme}
Zunächst eine kurze Erläuterung des Begriffs. Unter dem Titel "Scham. Lautloses Tönen" erschien 2012 ein Themenheft der Musiktherapeutischen Umschau (MU). Hierin beschrieb Tilman Weber in knapper Form und aus unterschiedlichen theoretischen Perspektiven die "Scham" folgendermaßen:
"Vielleicht kann man sagen, dass Scham aus triebpsychologischer Sicht einem typischen Es - Überich - Konflikt entspringt, aus objektpsychologischer Sicht der Diskrepanz zwischen Individuum und Gesellschaft und aus ichpsychologischer Sicht der Differenz zwischen Anspruch und Vermögen" \autocite[215]{weber2012}. 

Tiedemann, tiefenpsychologischer und analytischer Psychotherapeut und Forscher, beschreibt Scham als einen Vorgang, in dem "das Subjekt eine Infragestellung und Bedrohung der sozialen Wertschätzung, Akzeptanz und Anerkennung" \autocite[219]{tiedemann2012} erlebt. Daher tritt sie meist in solchen Situationen auf, in denen ein "Mensch etwas Intimes preisgibt" \autocite[219]{tiedemann2012}. 

Die Stimme ist das intimste körpereigene Instrument des Menschen. Jede stimmliche Äußerung gibt ein Stück unserer individuellen Ge-stimmt-heit preis. An unterschiedlichen Stellen in der hierzu relevanten Literatur wurde immer wieder darauf hingewiesen, dass eine stimmliche Maskierung des eigenen affektiven Zustandes nicht möglich ist \autocite[vgl.][279]{decker-voigt1992} \autocite[vgl.][481]{rittner2009a}. Je nach Stellung der Stimmknörpelchen und des Tonus der über 100 an der stimmlichen Äußerung beteiligten Muskeln \autocite[vgl.][40]{cramer1998} gestaltet sich das nach außen Dringende. Denn jegliches Zurückhalten oder Verdecken von aufkeimenden Emotionen führt zu Störungen der Feinabstimmung innerhalb des Stimmapparates. Dies kann zu Muskelverspannungen führen, die wiederum in der erklingenden Stimme durch einen harten Stimmansatz, verhauchen der Stimme durch nicht zu Klang gewordener herausströmender Luft, eine höhere oder tiefere Stimmfrequenz u.a. zu hören sind \autocite[vgl.][279]{decker-voigt1992}. 
Laut Rittner vernehmen wir die "emotionale Tönung" der Person durch die Prosodie [=das, was dazu singt). Insbesondere durch den Klang der Vokale, welche physikalisch betrachtet als Klangträger gleichmäßige Schwingungsmuster, Konsonanten hingegen als Geräuschträger ungleichmäßige Schwingungswellen erzeugen, erfahren wir im Sinne der oben genannten physiologischen Vorgänge etwas über die Gestimmtheit unseres Gegenübers. Dabei kommt dem Singen eine besondere Rolle zu: "Singen unterscheidet sich vom Sprechen dadurch, dass der klanglich-nonverbale Schwingungsanteil der emotionalen Botschaft durch die Verlängerung der Vokale im ausgedehnten Ausatem deutlich in den Vordergrund tritt" \autocite[205]{rittner2008}.

Die vorherigen Ausführungen machen deutlich, dass es sich bei der gesanglichen Äußerung um ein sehr unmittelbares und intimes Zeigen der emotionalen Verfassung handelt. Die stimmliche Äußerung ist direkter als das Spielen eines Musikinstruments, welches bereits räumlich von uns getrennt ist und evtl. sogar einen Schutz bietet i.S. eines sich dahinter Verstecken-Könnens. Wenn wir also im musiktherapeutischen Kontext mit der Singstimme arbeiten, sollten wir uns dieser unmittelbaren Verbindung bewusst sein. Denn das Singen, vorallem in der Improvisation, welche häufig mit der Sorge um Kontrollverlust verbunden ist, kann als "zu starker emotionaler Ausdruck abgewehrt und blockiert werden" \autocite[485]{rittner2009a}. Um jedoch der Abwehr aus Scham entgegen zu wirken, sei es laut Rittner wichtig, stimmliche Interventionen im Verlauf der musiktherapeutischen Behandlung methodisch anzubahnen. So könne beispielsweise mit bekannten, durchkomponierten Liedern, Melodien und Rhythmen begonnen werden, um sich der ursprünglichen stimmlichen, improvisierenden Äußerung, wie sie zu Beginn des Lebens wertfrei bestand, zu nähern. 

\subsection{Diagnostische Überlegungen}
Im Rahmen der Diagnostik gibt die Stimme des Patienten bereits wichtige Hinweise auf dessen aktuelle Befindlichkeit sowie seine Stimmung. Durch aufmerksames Hinhören kann erspürt werden, ob das Gesagte in sich "stimmig", sprich kongruent ist. Zudem gibt sie bereits Auskunft über den biografischen Hintergrund der sich äußernden Person, denn sie hat sich mit unseren über die Zeit gesammelten Lebenserfahrungen weiterentwickelt und diese in sich aufgesogen. So gilt die Stimme unter Experten als "Klingendes Hologramm der Persönlichkeit" \autocite{adamek1999}, "lauthafte Biographie" \autocite{gundermann1994} und bei Rittner erfahren wir über den Klang der Stimme etwas über die "leib-seelische Gewordenheit" \autocite[211]{rittner2008} des Menschen.
Dabei scheint jedoch nicht so sehr der Inhalt des Gesagten aufschlussreich, sondern vielmehr der "Klang der Stimme (Klangspektrum, Modulation, Lautstärkeänderungen, Stimmsitz, Vokaleinsatz etc.), die Sprechweise (Artikulation, Phrasierung, Pausensetzung etc.) und die Atmung (Atemfrequenz, Atemtiefe, hörbares Einatmen, Sitz des Atemraumes im Körper etc.)" als auch "die Art der Gestaltung von Sprechpausen, Abbrüchen, Momenten des Innehaltens, Verzögerungen u.ä." \autocite[210]{rittner2008}. Diese stillen Momente vertiefen jedoch oftmals das atmosphärische Wahrnehmen und können zu einem tieferen Verstehen ohne Ablenkung führen. 
Gerade in Momenten der Inkongruenz von Semantik, Prosodie, Mimik, Gestik etc. wurde festgestellt, dass der Zuhörer sich in diesem Falle weniger auf den Inhalt des Gesprochenen seines Gegenübers bezieht, sondern in einen visuell-auditiven Modus wechselt, in welchem die Semantik in den Hintergrund rückt \autocite [vgl.][206f.]{rittner2008}.

\subsection{Weitere Aspekte musiktherapeutischen Arbeitens mit der Stimme}
Rittner

\subsection{Vokalimprovisation}
Die Vokalimprovisation ist eine sehr spielerische Möglichkeit, die Stimme in die musiktherapeutische Arbeit einzubeziehen. Sie gilt als der "Einsatz des gesamten Ausdrucksspektrums stimmhafter wie stimmloser Äußerungen, meist im Schutz einer Gruppe mit oder ohne Themenvorgabe" \autocite [108]{rittner1990}. Dabei sei sie stets gebunden an persönliche, situative, musikalisch oder symbolische Inhalte und verfolge therapeutische, pädagogische oder musikalisch-künstlerische Ziele \autocite [109]{rittner1990}.


\subsection{Die Wichtigkeit des Einbezugs von Körper und Atem in dieser Arbeitsweise}
Ilse Middendorf
Schlaffhorst Andersen
MT und Atem (Buch)
Annette Cramer
Rittner
Hertha Richter

\section{Die Wirkung des Singens auf Körper und Psyche} 
Bossinger, Adamek und Cramer
Selbstheilungskräfte





\newpage\thispagestyle{empty}
% ----------------------------------------------------------------------